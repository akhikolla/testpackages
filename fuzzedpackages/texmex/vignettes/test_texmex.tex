\documentclass[a4paper]{article}\usepackage[]{graphicx}\usepackage[]{color}
% maxwidth is the original width if it is less than linewidth
% otherwise use linewidth (to make sure the graphics do not exceed the margin)
\makeatletter
\def\maxwidth{ %
  \ifdim\Gin@nat@width>\linewidth
    \linewidth
  \else
    \Gin@nat@width
  \fi
}
\makeatother

\definecolor{fgcolor}{rgb}{0.345, 0.345, 0.345}
\newcommand{\hlnum}[1]{\textcolor[rgb]{0.686,0.059,0.569}{#1}}%
\newcommand{\hlstr}[1]{\textcolor[rgb]{0.192,0.494,0.8}{#1}}%
\newcommand{\hlcom}[1]{\textcolor[rgb]{0.678,0.584,0.686}{\textit{#1}}}%
\newcommand{\hlopt}[1]{\textcolor[rgb]{0,0,0}{#1}}%
\newcommand{\hlstd}[1]{\textcolor[rgb]{0.345,0.345,0.345}{#1}}%
\newcommand{\hlkwa}[1]{\textcolor[rgb]{0.161,0.373,0.58}{\textbf{#1}}}%
\newcommand{\hlkwb}[1]{\textcolor[rgb]{0.69,0.353,0.396}{#1}}%
\newcommand{\hlkwc}[1]{\textcolor[rgb]{0.333,0.667,0.333}{#1}}%
\newcommand{\hlkwd}[1]{\textcolor[rgb]{0.737,0.353,0.396}{\textbf{#1}}}%
\let\hlipl\hlkwb

\usepackage{framed}
\makeatletter
\newenvironment{kframe}{%
 \def\at@end@of@kframe{}%
 \ifinner\ifhmode%
  \def\at@end@of@kframe{\end{minipage}}%
  \begin{minipage}{\columnwidth}%
 \fi\fi%
 \def\FrameCommand##1{\hskip\@totalleftmargin \hskip-\fboxsep
 \colorbox{shadecolor}{##1}\hskip-\fboxsep
     % There is no \\@totalrightmargin, so:
     \hskip-\linewidth \hskip-\@totalleftmargin \hskip\columnwidth}%
 \MakeFramed {\advance\hsize-\width
   \@totalleftmargin\z@ \linewidth\hsize
   \@setminipage}}%
 {\par\unskip\endMakeFramed%
 \at@end@of@kframe}
\makeatother

\definecolor{shadecolor}{rgb}{.97, .97, .97}
\definecolor{messagecolor}{rgb}{0, 0, 0}
\definecolor{warningcolor}{rgb}{1, 0, 1}
\definecolor{errorcolor}{rgb}{1, 0, 0}
\newenvironment{knitrout}{}{} % an empty environment to be redefined in TeX

\usepackage{alltt}
\usepackage[british]{babel}
\usepackage{fancyhdr}
\usepackage{rotating}
\usepackage{morefloats}

%\VignetteIndexEntry{test_texmex}
%\VignetteEngine{knitr}
\IfFileExists{upquote.sty}{\usepackage{upquote}}{}
\begin{document}
\title{A validation suite for texmex}
\author{Janet E. Heffernan and Harry Southworth}
\maketitle%\thispagestyle{empty}
\tableofcontents
%\clearpage



\section{Introduction}
The texmex~\cite{texmex} package for R provides a suite of test routines
These tests include comparisons of
parameter estimates with published versions -- particulary values published in
works by Coles~\cite{coles} and Heffernan and Tawn~\cite{heffernanTawn} -- as well
as many comparisons with values produced by independently written code, and many
logical checks and checks for routine failure in the context of inappropriate
usage.

Many graphs are produced by the test suite and these can be visually compared
with published versions, or evaluated against known plausible output.

The {\tt testthat}~\cite{testthat} package is used to provide the testing environment.

In total, over 1000 tests are performed providing a reasonable degree of
confidence that {\tt texmex} produces the output expected of it.

\subsection{Software}
R version 3.6.2 (2019-12-12)~\cite{R} was used in the construction of this
vignette.

\subsection{Acknowledgements}
Some of the independently written code that is used for validation is borrowed
from the {\tt ismev}~\cite{coles} and {\tt evd}~\cite{evd} packages. Other code has been provided by Yiannis Papastathopoulos~\cite{yiannis} and by Paul Metcalfe of
AstraZeneca.

The development of the {\tt texmex} package, including its test suite, was
partially funded by AstraZeneca.

\section{Using the validation suite}
To install the test suite, it is necessary to install the package with the option
{\tt --install-tests}. For example, if installing version 2.3 at the command
line from the package source (the .tar.gz file), the command would be

\begin{verbatim}
R CMD INSTALL --install-tests texmex_2.3.tar.gz
\end{verbatim}

The test suite depends upon the {\tt devtools}~\cite{devtools} package, so it is
necessary to load that package before attempting to run any tests.

\begin{knitrout}
\definecolor{shadecolor}{rgb}{0.969, 0.969, 0.969}\color{fgcolor}\begin{kframe}
\begin{alltt}
\hlkwd{library}\hlstd{(texmex)}
\hlkwd{library}\hlstd{(devtools)}
\end{alltt}
\end{kframe}
\end{knitrout}

The test scripts are located under the package directory in {\tt tests/testthat}.
To run a specific test script, for example the test for plotting of return levels from evm fitting, you can use

\begin{knitrout}
\definecolor{shadecolor}{rgb}{0.969, 0.969, 0.969}\color{fgcolor}\begin{kframe}
\begin{alltt}
\hlcom{#devtools::test("../../texmex",filter="plotrl.evm")}
\hlkwd{test_dir}\hlstd{(}\hlstr{"../tests/testthat"}\hlstd{,} \hlkwc{filter}\hlstd{=}\hlstr{"plotrl.evm"}\hlstd{)}
\end{alltt}
\begin{verbatim}
## v |  OK F W S | Context
## 
/ |   0       | plotrl.evm
v |   0     1 | plotrl.evm
## --------------------------------------------------------------------------------
## test-plotrl.evm.R:4: skip: plotrl.evm behaves as it should
## Reason: On CRAN
## --------------------------------------------------------------------------------
## 
## == Results =====================================================================
## OK:       0
## Failed:   0
## Warnings: 0
## Skipped:  1
\end{verbatim}
\end{kframe}
\end{knitrout}

inserting your own path to the package.  The plots produced can be checked against the published figures which are indicated in the plot titles.  Most tests carry out numerical tests rather than graphical ones, as in the following set of tests for calculations of the GPD distribution function in a range of contexts:

\begin{knitrout}
\definecolor{shadecolor}{rgb}{0.969, 0.969, 0.969}\color{fgcolor}\begin{kframe}
\begin{alltt}
\hlcom{#devtools::test("../../texmex",filter="pgpd")}
\hlkwd{test_dir}\hlstd{(}\hlstr{"../tests/testthat"}\hlstd{,} \hlkwc{filter}\hlstd{=}\hlstr{"pgpd"}\hlstd{)}
\end{alltt}
\begin{verbatim}
## v |  OK F W S | Context
## 
/ |   0       | pgpd
x |   0 1     | pgpd
## --------------------------------------------------------------------------------
## test-pgpd.R:4: error: pgpd behaves as it should
## object '.evd.pgpd' not found
## --------------------------------------------------------------------------------
## 
## == Results =====================================================================
## OK:       0
## Failed:   1
## Warnings: 0
## Skipped:  0
\end{verbatim}
\end{kframe}
\end{knitrout}


To run all test scripts (which will take a while), the command
is as follows.  The output is not shown here.

\begin{knitrout}
\definecolor{shadecolor}{rgb}{0.969, 0.969, 0.969}\color{fgcolor}\begin{kframe}
\begin{alltt}
\hlstd{devtools}\hlopt{::}\hlkwd{test}\hlstd{(}\hlstr{"../../texmex"}\hlstd{)}
\end{alltt}
\end{kframe}
\end{knitrout}

\bibliography{texmex}
\bibliographystyle{plain}
\end{document}
