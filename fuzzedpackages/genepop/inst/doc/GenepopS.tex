\documentclass[12pt,]{book}
\usepackage{lmodern}
\usepackage{amssymb,amsmath}
\usepackage{ifxetex,ifluatex}
\usepackage{fixltx2e} % provides \textsubscript
\ifnum 0\ifxetex 1\fi\ifluatex 1\fi=0 % if pdftex
  \usepackage[T1]{fontenc}
  \usepackage[utf8]{inputenc}
\else % if luatex or xelatex
  \ifxetex
    \usepackage{mathspec}
  \else
    \usepackage{fontspec}
  \fi
  \defaultfontfeatures{Ligatures=TeX,Scale=MatchLowercase}
\fi
% use upquote if available, for straight quotes in verbatim environments
\IfFileExists{upquote.sty}{\usepackage{upquote}}{}
% use microtype if available
\IfFileExists{microtype.sty}{%
\usepackage{microtype}
\UseMicrotypeSet[protrusion]{basicmath} % disable protrusion for tt fonts
}{}
\usepackage[margin=1in]{geometry}
\usepackage{hyperref}
\hypersetup{unicode=true,
            pdftitle={Genepop version 4.7.5},
            pdfauthor={F. Rousset},
            pdfborder={0 0 0},
            breaklinks=true}
\urlstyle{same}  % don't use monospace font for urls
\usepackage{longtable,booktabs}
\usepackage{graphicx,grffile}
\makeatletter
\def\maxwidth{\ifdim\Gin@nat@width>\linewidth\linewidth\else\Gin@nat@width\fi}
\def\maxheight{\ifdim\Gin@nat@height>\textheight\textheight\else\Gin@nat@height\fi}
\makeatother
% Scale images if necessary, so that they will not overflow the page
% margins by default, and it is still possible to overwrite the defaults
% using explicit options in \includegraphics[width, height, ...]{}
\setkeys{Gin}{width=\maxwidth,height=\maxheight,keepaspectratio}
\IfFileExists{parskip.sty}{%
\usepackage{parskip}
}{% else
\setlength{\parindent}{0pt}
\setlength{\parskip}{6pt plus 2pt minus 1pt}
}
\setlength{\emergencystretch}{3em}  % prevent overfull lines
\providecommand{\tightlist}{%
  \setlength{\itemsep}{0pt}\setlength{\parskip}{0pt}}
\setcounter{secnumdepth}{5}
% Redefines (sub)paragraphs to behave more like sections
\ifx\paragraph\undefined\else
\let\oldparagraph\paragraph
\renewcommand{\paragraph}[1]{\oldparagraph{#1}\mbox{}}
\fi
\ifx\subparagraph\undefined\else
\let\oldsubparagraph\subparagraph
\renewcommand{\subparagraph}[1]{\oldsubparagraph{#1}\mbox{}}
\fi

%%% Use protect on footnotes to avoid problems with footnotes in titles
\let\rmarkdownfootnote\footnote%
\def\footnote{\protect\rmarkdownfootnote}

%%% Change title format to be more compact
\usepackage{titling}

% Create subtitle command for use in maketitle
\newcommand{\subtitle}[1]{
  \posttitle{
    \begin{center}\large#1\end{center}
    }
}

\setlength{\droptitle}{-2em}

  \title{Genepop version 4.7.5}
    \pretitle{\vspace{\droptitle}\centering\huge}
  \posttitle{\par}
    \author{F. Rousset}
    \preauthor{\centering\large\emph}
  \postauthor{\par}
      \predate{\centering\large\emph}
  \postdate{\par}
    \date{This documentation: 23 February 2020}

\usepackage{makeidx}
\usepackage[utf8]{luainputenc}
\makeindex
\newcommand{\Rst}{\ensuremath{R_\mathrm{ST}}}
\newcommand{\Fst}{\ensuremath{F_\mathrm{ST}}}
\newcommand{\Fis}{\ensuremath{F_\mathrm{IS}}}
\newcommand{\Fit}{\ensuremath{F_\mathrm{IT}}}
\newcommand{\Genepop}[1]{\textsc{Genepop}1}
\newcommand{\Migraine}{\textsc{Migraine}}
\newcommand{\optN}[2]{1}
\newcommand{\afaire}[1]{\textcolor{red}{\textbf{1}}}
\newcommand{\webpage}{\href{http://kimura.univ-montp2.fr/~rousset/Genepop.htm}{Genepop}}

\begin{document}
\maketitle

{
\setcounter{tocdepth}{1}
\tableofcontents
}
\chapter{Introduction}\label{introduction}

\section{Purpose}\label{purpose}

This is a documentation for the Genepop software, distributed both as
stand-alone software and as an R package. Genepop implements a mixture
of traditional methods and some more focused developments:

\begin{itemize}
\item
  \textbf{It computes exact tests} for Hardy-Weinberg equilibrium, for
  population differentiation and for genotypic disequilibrium among
  pairs of loci;
\item
  \textbf{It computes estimates} of \(F\)-statistics, null allele
  frequencies, allele size-based statistics for microsatellites, etc.,
  and of number of immigrants by Barton \& Slatkin's 1986 private allele
  method;
\item
  \textbf{It performs analyses of isolation by distance} from pairwise
  comparisons of individuals or population samples, including confidence
  intervals for ``neighborhood size''.
\end{itemize}

A formal reference for the current version of Genepop is
\href{http://dx.doi.org/10.1111/j.1471-8286.2007.01931.x}{Rousset
(2008)}. Likelihood methods based on coalescent algorithms are being
developed in a distinct software, Migraine (Rousset and Leblois 2007;
Rousset and Leblois 2012; Leblois et al. 2014).

Genepop also converts data from the Genepop input format to formats of
some softwares that were around in Genepop's youth (Raymond and Rousset
1995b); there has been little need to update this option as many more
recent softwares for population genetic analyses read input files in the
Genepop format.

\section{The two Genepop
distributions}\label{the-two-genepop-distributions}

Genepop is now distributed both as an R package, and as stand-alone
software. See the
\href{http://kimura.univ-montp2.fr/~rousset/Genepop.htm}{Genepop
distribution page} for the latter. This documentation describes the use
of the executable. The functionalities it describes are available in an
R session, using R functions described only in the package
documentation.

\section{Changes since version 4.0}\label{changes-since-version-4.0}

\index{Genepop@\Genepop, differences from previous versions}

\textbf{Version 4.7.2}

A new keyword \texttt{intra\_all\_types} for setting
``popTypeSelection'' allows one to perform a single spatial regression
(but not Mantel tests) for all pairs of individuals or populations
within types (e.g., individuals within patches, excluding pairwise
statistics for pairs of individuals between patches).

Yet another problem has been fixed for Mantel tests' handling of missing
pairwise genetic information (specifically for pairs of ``pop'' -- most
likely, individuals -- sharing no genotypic information at any locus).

\textbf{Version 4.7.1}

Genepop is now also distributed as an R package. It now uses the
implementation of the Mersenne twister pseudo-random number generator
found in recent C++ compilers. This has two implications. First, a
recent compiler must be used, as described below. Second, test results
of previous versions cannot be exactly replicated.

The format of a few file outputs has been modified (in particular the
reporting of extreme values of some global tests).

\textbf{Version 4.6}

A bootstrap analysis of mean differentiation has been introduced, in
particular to allow comparison of the mean differentiation observed over
a given range of geographical distances, in intra vs.~inter-ecotypic
analyses. It can be called by the setting
\texttt{meanDifferentiationTest}.

The Mantel test based on regression slope (not the one on ranks) was not
handling appropriately cases where some pairwise data had to be
excluded. This is corrected. Such cases concern in particular pairs of
samples in the same location (e.g., pairs of individuals), when
geographical distance is log-transformed, because the pairwise
differentiation between such individuals cannot be used for the
computation of the regression. The bootstrap analyses were already
handling correctly this case.

\textbf{Version 4.5}

A new keyword \texttt{inter\_all\_types} for setting
``popTypeSelection'' allows one to perform spatial regressions (but not
Mantel tests) between all pairs of individuals or populations belonging
to different types (e.g., individuals belonging to different patches,
excluding pairwise statistics for pairs of individuals within patches).

\textbf{Version 4.4}

Mantel tests are by default no longer based on rank correlation. The
older rank tests can be performed using the new \texttt{MantelRankTest}
setting. In addition, a \texttt{MaximalDistance} setting has been added,
affecting the computation of spatial regressions.

\textbf{Version 4.3}

Two new ``miscellaneous'' conversion options have been added: option 8.5
converts population data to individual data (as 8.4) but keeps the
individual names (hence the geographic location of each individual); and
option 8.6 randomly samples haploid data at diploid loci.

\textbf{Version 4.2}

One can now perform all isolation-by-distance analyses with a
user-provided distance matrix instead of the geographic distance matrix
computed from the coordinates of the samples (\texttt{geoDistFile}
setting).

\textbf{Version 4.1}

It is possible to test trends in gene diversity among samples.

Analyses of isolation by distance have been strengthened in several
ways. Variants of previously described estimators have been implemented
for both haploid and diploid data. 0ne can select subsets of the data
for analyses of isolation by distance within and between these subsets.
Further, analysis of isolation by distance from several one-locus
genetic distance matrices is now possible through the
\texttt{MultiMigFile} option. In contrast to \texttt{IsolationFile},
this allows the construction of bootstrap confidence intervals. Finally,
it is possible to test specific values of the slope of the spatial
regression, using the \texttt{testPoint} setting.

The input file reading procedure is better protected against nonstandard
file formats (in particular those produced by some Microsoft software
under Mac OS X).

The new sub-option 8.4 has been added to convert population-based data
to individual-based data (each individual in its own \texttt{Pop}).

\textbf{Version 4.0}

Version 4.0 was a complete rewrite of the fossil version 3.4, with the
following changes:

Use of the \(G\) (log likelihood ratio) statistic has been generalized
to all contingency tables (though previous probability tests implemented
in Genepop are still available). Genepop now provides bootstrap
confidence intervals for strength of isolation by distance between
groups of individuals, an alternative estimator for analyses of
``differentiation between individuals'', and facilities to evaluate the
performance of these methods. The genetic distance matrix produced by
these options can also be exported in Phylip (Felsenstein 2005) format.
The option for null allele estimation implements additional estimators
with confidence intervals, and its output is better organized.

Some \textbf{additional facilities} have been implemented for better
ease of use. Earlier versions of Genepop required from the user some
effort to deal with either 3-digits-coded \index{Allele coding!3-digits}
alleles or with haploid data. Genepop is more practical, in that haploid
\index{Haploid data} and diploid genotypes in both 2- or 3-digits allele
codings are automatically recognized as such by the program and all
these different types of data can be mixed in the same input file. The
input format is otherwise unchanged so that \textbf{input files prepared
for earlier versions of Genepop are still read by Genepop} (backward
compatibility).

In addition, Genepop's behaviour can be controlled using an option file
and by inline arguments in a console command line. This allows batch
calls to Genepop and repetitive use of Genepop on simulated data.
However, those familiar with the old Genepop menus can also use Genepop
in an almost unchanged way.

Previous Genepop distributions included two small utilities, hw.bat
\index{HW program} and struc.bat, \index{Struc program} for testing of
single data matrices using a fast ad hoc data input. These facilities
are available in Genepop 4.0 through the \texttt{HWfile}
\index{HWfile setting} and \texttt{StrucFile} options.
\index{StrucFile setting} Previous Genepop distributions also included
the Isolde \index{Isolde program} program for analysis of isolation by
distance between groups of individuals, from one genetic distance and
one geographic distance matrices. All such analyses can now be performed
through the unique Genepop executable (other facilities that were unique
to Isolde are now accessible through the \texttt{IsolationFile}
setting).

Other minor, and often trivial, differences with earlier versions of
Genepop will be pointed out in footnotes.

\chapter{Installing Genepop and session
examples}\label{installing-genepop-and-session-examples}

\section{Installation}\label{installation}

\subsection{R package}\label{r-package}

As any R package, it can be installed by
\texttt{install.packages("genepop")} if on CRAN, and more generally by
\texttt{install.packages(}\texttt{,type="source",repos=NULL)}. See the R
documentation for more information.

\subsection{Stand-alone executable}\label{stand-alone-executable}

Under \textbf{Microsoft
Windows}\index{Microsoft Windows!installation on}, one only needs to
unzip/copy the executable on hard disk. Both 32- and 64-bit versions of
the executables are distributed. Under \textbf{Linux/Mac
OSX}\index{Linux}, \index{Linux!installation on} extract all c++ sources
from the distributed \texttt{sources.tar.gz} (or from the \texttt{src/}
subdirectory of the R package sources, \textbf{except}
\texttt{RcppExports.cpp}), and compile with a compiler that supports the
C++11 standards. For \textbf{Windows}, one can use \texttt{g++} version
4.9.3 (distributed with recent versions of the R tools) with an ad hoc
flag:

\texttt{g++\ -std=c++11\ -o\ Genepop\ *.cpp\ -O3}

(\texttt{O} in \texttt{-O3} is the letter O, not zero). With more recent
versions of \texttt{g++} (\textgreater{}=6.0) or \texttt{clang++}, no
such flag is required:

\texttt{g++\ -o\ Genepop\ *.cpp\ -O3}.

The data files do not need to be in the same directory as the
executable\footnote{\ldots{}in contrast to earlier versions of Genepop.};
however, users might find that specifying path names under Windows is
not as easy at it should.

Examples and documentation files are included in the R package and are
available on the
\href{http://kimura.univ-montp2.fr/~rousset/Genepop.htm}{Genepop
distribution page}.

Linkdos\index{Linkdos program}, a program described by Garnier-Géré and
Dillmann (1992), is distributed with (but is not part of) Genepop. It is
originally a DOS program, but the source file distributed can be
recompiled under Linux using the Free Pascal compiler (or at least
``could'', since this is no longer maintained/checked).

\section{Example sessions}\label{example-sessions}

To reproduce the examples of this session one should copy in a personal
directory the examples files found in the \texttt{extdata/} subdirectory
of the packageor on the
\href{http://kimura.univ-montp2.fr/~rousset/Genepop.htm}{Genepop
distribution page}.

\subsection{Example 1: basic session}\label{example-1-basic-session}

Open a console window in the directory where Genepop has been installed
and just execute

\begin{verbatim}
 Genepop
\end{verbatim}

If Genepop has never been run before, it will ask for an input file.
Otherwise, the main menu should appear, in which case you should use the
\texttt{C} option to load this input file. For this sample session, the
file name to be given is \texttt{sample.txt}. Genepop will display some
information about the file read, then display the main menu:

\begin{verbatim}
-------> Change Data ................... C


Testing :
    Hardy-Weinberg exact tests (several options) ...................... 1
    Exact tests for genotypic disequilibrium (several options) ........ 2
    Exact tests for population differentiation (several options) ...... 3

Estimating:
    Nm estimates (private allele method) .............................. 4
    Allele frequencies, various Fis and gene diversities .............. 5
    Fst & other correlations, isolation by distance (several options).. 6

Ecumenicism and various utilities:
    Ecumenicism: file conversion (several options) .................... 7
    Null alleles and miscellaneous input file utilities ............... 8

QUIT Genepop .......................................................... 9

Your choice? :
\end{verbatim}

Each option will be described later. Let us see some tests for
heterozygote deficiency. Reply \texttt{1}, next \texttt{1}, next
\texttt{y}(es). As indicated, the results of the analysis are stored in
the file \texttt{sample.txt.D}.

The next example illustrates a slightly more elaborate use of Genepop.

\subsection{Example 2: using the settings
file}\label{example-2-using-the-settings-file}

Execute

\begin{verbatim}
 Genepop settingsFile=SampleSettings.txt
\end{verbatim}

\index{SettingsFile setting}Do not add spaces in the arguments.
Capitalisation matters for file names (here \texttt{SampleSettings.txt})
if it matters for the operating system (i.e.~for Linux).

You can see that the previous and additional analyses are performed, and
that you just need to hit Return each time Genepop stops and waits for
feedback. Finally, you are brought back to the main menu. Simple
instructions for performing the analyses are contained in the
\texttt{SampleSettings.txt} file, which you may edit. Section
\ref{sec-settings} will explain how to use this file. By default,
Genepop seeks and eventually reads instructions in a
\texttt{Genepop.txt} file. You can see that one such file is present and
was thus read when performing Example 1.

\subsection{Example 3: Batch
processing}\label{example-3-batch-processing}

Execute the same command as in the previous example but with one more
statement:

\begin{verbatim}
 Genepop settingsFile=SampleSettings.txt Mode=Batch
\end{verbatim}

\index{Mode setting} \index{Batch mode}Genepop should perform the same
computations as in the previous example but it will not stop and wait
for feedback, and will exit after completion of the computations. Note
again that spaces are not allowed within each of the arguments
\texttt{settingsFile=SampleSettings.txt} and \texttt{Mode=Batch}, nor
more generally in arguments specified on the command line.

The batch mode makes it easy to analyze multiple files. However, note
that concurrent Genepop processes\index{Concurrent processes} should be
run in distinct directories. Otherwise, the temporary files of each
process might conflict with each other.

\chapter{The input file}\label{the-input-file}

\index{Input format} As illustrated by the following examples, the input
format requested by Genepop is:

\begin{itemize}
\tightlist
\item
  \textbf{First line: anything} Use this line to store information about
  your data.
\item
  \textbf{Locus names} They may be given one per line, or on the same
  line but separated by commas. \texttt{Pop} sample indicator
  (Capitalization does not matter)\footnote{Earlier versions of Genepop
    only accepted \texttt{Pop}, \texttt{POP} and \texttt{pop}\ldots{}}.
  Each sample from a different geographical original is declared by a
  line with a \texttt{pop} statement.
\item
  \textbf{Information for first individual.} An example is:
  \texttt{ind\#001\ fem\ \ ,0101\ 0202\ 0000\ 0410} Here
  \texttt{ind\#001\ fem} is an identifier for your personal use. You can
  use any character (except a comma!). You may leave it blank (at least
  one space) if you wish. The last identifier of every sub-population is
  used by Genepop as the sample name in output files. The comma between
  the identifier and the list of genotypes is required. \texttt{0101}
  indicates that this individual is homozygous for the \texttt{01}
  allele at the first locus. The same individual is homozygous for the
  \texttt{02} allele at the second locus (\texttt{0202}). Data are
  missing at the third locus (\texttt{0000}). At the fourth locus, the
  genotype is \texttt{0410}, which indicates the presence of alleles
  \texttt{04} and \texttt{10}.
\item
  \textbf{More individuals} Each individual information starts on a new
  line, but may extend over several lines (do not start a new line in
  the middle of a one-locus genotype!).
\item
  \textbf{More samples} each declared by a \texttt{pop} statement on a
  new line
\item
  \textbf{Blank lines} at the end of the file are removed by Genepop.
\end{itemize}

An example of a short input file is given below:

\begin{verbatim}
 Title line: "Grape populations in southern France"
 ADH Locus 1
 ADH #2
 ADH three
 ADH-4
 ADH-5
 mtDNA
 Pop
 Grange des Peres  ,  0201 003003 0102 0302 1011 01
 Grange des Peres  ,  0202 003001 0102 0303 1111 01
 Grange des Peres  ,  0102 004001 0202 0102 1010 01
 Grange des Peres  ,  0103 002002 0101 0202 1011 01
 Grange des Peres  ,  0203 002004 0101 0102 1010 01
 POP
 Tertre Roteboeuf ,      0102 002002 0201 0405 0807 01
 Tertre Roteboeuf ,      0102 002001 0201 0405 0307 01
 Tertre Roteboeuf ,      0201 002003 0101 0505 0402 01
 Tertre Roteboeuf ,      0201 003003 0301 0303 0603 01
 Tertre Roteboeuf ,      0101 002001 0301 0505 0807 01
 pop
 Bonneau 01   , 0101    002002 0304 0805 0304 01
 Bonneau 02   , 0201    002002 0404 0505 0304 01
 Bonneau 03   , 0101    002100 0304 0505 0101 01
 Bonneau 04 , 0101    100100 0204 0805 0304 01
 Bonneau 05   , 0101    100002 0104 0808 0304 01
 Pop
  ,            0000 002001 0202 0402 0007 01
  ,            0200 002001 0202 0205 0707 01
  ,            0010 002001 0101
 0105 0807 01
 last pop,      0101 002001 0101 0401 0807 02
\end{verbatim}

This example shows some useful features of the input file:

\begin{itemize}
\item
  There is no constraint on the number of blanks separating the various
  fields.
\item
  The individual identifier has a free format.
\item
  Alleles are numbered from 01 to 99 or 001 to 999 if needed.
  \index{Allele coding} In 3-digits coding,
  \index{Allele coding!3-digits} (say) homozygotes for the \texttt{90}
  allele are noted \texttt{090090}, not \texttt{9090} as in the 2-digits
  format. 2-digits and 3-digits coding of alleles can be intermixed
  (among loci, not within loci!).\footnote{New to Genepop 4.0}
\item
  To designate alleles, consecutive numbers are not required.
\item
  haploid and diploid data\index{Haploid data} can be
  intermixed.\footnote{Also new to Genepop 4.0} 6-digits genotypes are
  recognized as 3-digits diploid
  genotypes;\index{Allele coding!3-digits} 4-digits genotypes are
  recognized as 2-digits diploid
  genotypes;\index{Allele coding!2-digits} 2- and 3-digits genotypes are
  recognized as haploid genotypes.\index{Haploid data} The same coding
  should be used consistently within each locus. See the
  \texttt{EstimationPloidy} setting for more information about analyzing
  haploid data. For haplo-diploid data\index{Haplo-diploid genotypes} at
  a given locus, the haploid genotypes should be coded as diploid
  genotypes with one unknown allele; note however that the information
  from haploid genotypes at haplo-diploid loci will be used only for
  genic contingency table tests, and will be ignored in estimation of
  genetic structure.
\item
  Genotypes can extend on more than one line (see penultimate
  individual)
\item
  To group various samples, just remove each relevant \texttt{Pop}
  separator.
\end{itemize}

It is possible to write all the locus names on one line, provided that a
comma is used as separator. This could be useful to clearly label each
column. Thus the above input file could have started as

\begin{verbatim}
 Title line: "Grape populations in southern France"
                      Loc1,Loc2,  ADH3,ADH4,ADH5,mtDNA
 Pop
 Grange des Peres  ,  0201 003003 0102 0302 1011 01
 ...
\end{verbatim}

Note the absence of comma after the last locus name.

There are however constraints to be obeyed

\begin{itemize}
\item
  Missing data\index{Missing data} should be indicated with \texttt{00}
  (or \texttt{000} for 3-digits coding) and not with blanks. The first
  locus in the last sample illustrates the various possibilities of
  missing data: no information (first individual coded \texttt{0000}) or
  partial information (only one allele is determined: allele \texttt{02}
  for the second individual coded \texttt{0200} and allele \texttt{10}
  for the third individual coded \texttt{0010}).
\item
  The number of locus names should correspond to the number of genotypes
  in each individual. If you remove one or several loci from your input
  file, you should remove both their names and the corresponding
  genotypes.
\item
  No empty line should be present in the data file.
\item
  Genepop accepts input file names either with the extension
  \texttt{.txt}\footnote{New to Genepop 4.0} or without any extension.
\item
  Genepop input files are ASCII text files.
\end{itemize}

The last point implies that under
\textbf{Windows},\index{Microsoft Windows!file format issues} you should
avoid using Microsoft Word to edit input files (and settings files as
well). Rather use a text editor such as
\href{http://notepad-plus-plus.org}{Notepad++}.\footnote{Other text
  editors including the Windows basic text editor may not show all
  end-of-line characters correctly.} It has also appeared that certain
Microsoft products under \textbf{Mac OS
X}\index{Mac OS X!file format issues} still produced files formatted
according to the older Mac format. Genepop now catches and corrects this
miserable feature.

One can also find some conversion tools (e.g.~from EXCEL) on the web.

If the input file is correctly read, the name of the larger allele
number is indicated for each locus. The number of distinct alleles for
each locus is provided upon request. If alleles have been labeled with
consecutive numbers from \texttt{01} onwards, then the name of the
larger allele will correspond to the number of distinct alleles for each
locus.

There are some limits to the number of samples and individuals imposed
by the compiler. These values, and a few other ones, are shown by
running ``\texttt{Genepop\ Maxima=}'' (see the \texttt{Maxima}
setting).\index{Sample size!limitations} However, these built-in maxima
are so large\footnote{in constrast to earlier versions of Genepop} as to
be practically infinite even in the era of whole-genome sequencing.
Computer memory, or user patience, are more likely limits.

\chapter{The settings file and command line
arguments}\label{sec-settings}

The settings file\index{Settings file} allows finer control of Genepop
and/or batch processing. Further control is possible by using optional
arguments when launching Genepop through the operating system command
line,\index{Command line} following the general syntax explained below
for the settings file, e.g.

\begin{verbatim}
 Genepop EstimationPloidy=Haploid DifferentiationTest=Proba
\end{verbatim}

Indeed, command line arguments are written in the file
\texttt{cmdline.txt}, then this file is read much as the settings
file.\footnote{Long command lines: under some old versions of Windows,
  the command line had a fairly limited maximum length, so it should
  have been used with moderation. This should no longer be a problem
  with recent versions of Windows, but who knows with Microsoft\ldots{}
  one may try to find more information about command-line string
  limitation on \texttt{support.microsoft.com}.}

Henceforth, menu options are called \emph{options} and batch
file/command line options are called \emph{settings}.

Running \texttt{Genepop\ help} will display the help information, which
so far is no more than a list of available settings, loosely grouped
semantically.\index{help} A file showing all possible settings is the
following:

\begin{verbatim}
 // sample Genepop settings file, showing all options.
 /*********** Syntax of this file:
 lines without 'equal' symbol are ignored (hence this one is).
 Lines beginning with a '/', /a '#' or a '%' are also ignored,
 even if they contain '=' (hence this one is).
 /*********** General options ***********
 Mode=Ask
 GenepopInputFile=sample.txt
 Dememorisation=10000
 BatchLength=5000
 BatchNumber=100
 //EstimationPloidy=Haploid
 //RandomSeed=12345678
 //MantelSeed=87654321
 /***     allele sizes stuff
 //AllelicDistance=Size
 AlleleSizes=1:5,2:10,3:15,10:50
 /*** selecting menu options
 MenuOptions=8
 /********** Option 1 (HW tests) ***********
 HWtests=Enumeration
 /           Emulating HW.BAT
 //HWFile=HWtest
 //HWfileOptions=4,3
 /********** Option 2 ("linkage" disequilibrium) ***********
 //          old Genepop behaviour
 /GameticDiseqTest=Proba
 /********** Option 3 (differentiation) ***********
 //          old Genepop behaviour
 /DifferentiationTest=Proba
 /           Emulating STRUC.BAT
 //strucFile=structest
 /********** Option 4 (private alleles) ***********
 //no specific setting, but may be affected
 //by the estimationPloidy setting
 /** Option 5 (basic information, Fis, gene diversities... )
 //no specific setting, but may be affected
 // by the AlleleSizes setting
 /***** Option 6 (F-statistics, isolation by distance) *****
 IsolationStatistic=e
 GeographicScale=Linear
 MinimalDistance=1
 CIcoverage=0.9
 testPoint=0.00123
 //MantelRankTest=
 /PopTypes= 1 2 1 2 3
 /PopTypeSelection= all
 //PhylipMatrix=
 /           Emulating ISOLDE
 //IsolationFile=Isoldetest
 /           Extending ISOLDE to multiple matrices
 //MultiMigFile=perlocusStuff
 / Isolation by distance with user-provided geographic distances
 //geoDistFile=someFile
 /********** Option 7 (file conversions) ***********
 //no specific setting
 /********** Option 8 (Various utilities) ***********
 NullAlleleMethod=ApparentNulls
 CIcoverage=0.9
 /******** Testing performance of some options *********
 // Option 6.x: options as above plus
 //Performance=aLinear
 //GenepopRootFile=file
 //JobMin=1
 //JobMax=100
 /********* Checking some limits of Genepop ***********
 //Maxima=
\end{verbatim}

Each setting is specified following a
\emph{Keyword}\texttt{=}\emph{value} syntax. Capitalisation is not
important (it is here only to ease reading) \emph{except} for file names
if the operating system cares about it (as Linux does).

By default, Genepop seeks settings in the file \texttt{Genepop.txt}, but
one can specify another settings file through the command line, as was
shown in the session examples:

\begin{verbatim}
 Genepop settingsFile=SampleSettings.txt
\end{verbatim}

The \texttt{SettingsFile}\index{SettingsFile} setting must be the first
argument on the command line.

Settings specific to each menu option will be explained along with the
description of each option. Settings affecting several menu options are
the following:

\texttt{GenepopInputFile}\index{GenepopInputFile} (or simply
\texttt{InputFile} \index{InputFile setting})

which is the name of the input file in Genepop format

\texttt{Dememorisation}\index{Dememorisation setting},
\texttt{BatchLength}\index{BatchLength setting} and
\texttt{BatchNumber}\index{BatchNumber setting}

\index{Markov chain algorithms!parameters}which are Markov Chain
parameters, which meaning is explained in Section
\ref{algorithms-for-exact-tests}:

\textbf{the dememorisation number} The default is 10000;\footnote{increased
  from Genepop 3.4's default} values below 100 are not allowed.

\textbf{the number of batches} The default is 20 for sub-options 1.4 and
1.5 (multisample HW tests), and 100 otherwise; values below 10 are not
allowed.

\textbf{the number of iterations per batch} The default is
5000;\footnote{increased from Genepop 3.4's default} values below 400
are not allowed.

The maximum allowed value of these parameters will depend on the C++
compiler (it is its maximum \texttt{size\_t}, that is at least 65535,
and typically much more on recent compilers). See the setting
\texttt{Maxima} if you really need more information about this value.

\texttt{EstimationPloidy}\index{EstimationPloidy}\index{Haploid data}

In multilocus estimates only diploid data are taken into account, unless
the setting \texttt{EstimationPloidy=Haploid} is given, in which case
only haploid data are taken into account. This setting applies to
options 4 (private allele method), 5.2 and 5.3 (for multilocus estimates
of gene diversities), and 6 (\(F\)-statistics and isolation by
distance).

\texttt{Mode}\index{Mode setting}

Genepop has three modes: \texttt{Mode=Ask} will ask for some feedback
even in cases where the answer has been prespecified (e.g.~through some
setting; this may be useful when one wishes to change some settings in
the course of a session). For example it will ask for confirmation of
the MC parameters. \texttt{Mode=Batch} will not wait for feedback:
execution of Genepop should complete without any user intervention. The
third mode, \texttt{Mode=Default} (which in most cases does not need to
be explicitly specified) will ask for unspecified settings but not
request confirmation of prespecified ones, and will also pause and wait
for feedback when some notable information is displayed.

\texttt{MenuOptions}\index{MenuOptions setting}

This tells Genepop to run the analyses as given through the menus:
\texttt{MenuOptions=1.1} will run option 1 sub-option 1 (test for
heterozygote deficit), \texttt{MenuOptions=1.1,2.2} will run option 1.1
then 2.2, and so on.

\texttt{AllelicDistance=Size}\index{AllelicDistance setting} (or
\texttt{=AlleleSize})

This tells Genepop to use allele size-based
statistics\index{Allele size-based statistics} (where meaningful).
Allele sizes are allele names unless specified by the next setting:

\texttt{AlleleSizes}\index{AlleleSizes setting}

In the above example, the first such line
\texttt{AlleleSizes=1:5,2:10,3:15,10:50} says that at the first locus,
allele 1 has size 5, allele 2 has size 10\ldots{} 0 cannot be given a
size since it means missing information. Any unlisted allele retain its
name as its size. The second line specifies allele size at the second
locus. The third line \texttt{AlleleSizes=} implies that at the third
locus, all alleles retain their name as their size (don't forget the
`\texttt{=}'). It is needed only so that the next line
\texttt{AlleleSizes=1:5,2:10,3:15,10:50} refers to the fourth locus. As
there are four \texttt{AlleleSizes} declarations, alleles retain their
name as their size for any locus beyond the fourth one.

\texttt{RandomSeed}\index{RandomSeed setting} and
\texttt{MantelSeed}\index{MantelSeed setting}
\index{Pseudo-random numbers} One may change the seed of the
pseudo-random number generator by the setting
\texttt{RandomSeed=}\emph{value}, except for the Mantel test for which
the seed is given by the setting \texttt{MantelSeed=}\emph{value}. The
default value for both seeds is 67144630.

\texttt{Maxima}\index{Maxima setting}

With this setting, Genepop will only display some maximal values,
including the maximum \texttt{int} and \texttt{long\ int} values for the
compiler (the Markov chain dememorization and batch length are
\texttt{long\ int} and the number of batches is \texttt{int}).

\chapter{All menu options}\label{all-menu-options}

\section{Option 1: Hardy-Weinberg (HW) exact
tests}\label{option-1-hardy-weinberg-hw-exact-tests}

The following menu appears:

\begin{verbatim}
Hardy Weinberg tests:

HW test for each locus in each population:
   H1 = Heterozygote deficiency.......1
   H1 = Heterozygote excess...........2
   Probability test...................3

Global test:
   H1 = Heterozygote deficiency.......4
   H1 = Heterozygote excess...........5

Main menu.............................6
\end{verbatim}

\subsection{Sub-options 1--3: Tests for each locus in each
population}\label{sub-options-13-tests-for-each-locus-in-each-population}

Three distinct tests are available, all concerned with the same null
hypothesis (random union of gametes). The difference between them is the
construction of the rejection zone. For the Probability test (sub-option
3), the probability of the observed sample is used to define the
rejection zone, and the \(P\)-value of the test corresponds to the sum
of the probabilities of all tables (with the same allelic counts) with
the same or lower probability. This is the ``exact HW test'' of Haldane
(1954), Weir (1996), Guo and Thompson (1992) and others. When the
alternative hypothesis of interest is heterozygote excess or deficiency,
more powerful tests than the probability test can be used (Rousset and
Raymond 1995). One of them, the score
test\index{Hardy-Wein\-berg tests!score test} or \(U\) test, is
available here, either for heterozygote deficiency (sub-option 1) or
heterozygote excess (sub-option 2). The multi-samples versions of these
two tests are accessible through sub-options 4 or 5.

Two distinct algorithms are available: first, the complete enumeration
method, as described by Louis and Dempster (1987). This algorithm works
for less than five alleles. As an exact \(P\)-value is calculated by
complete enumeration, no standard error is computed. Second, a Markov
chain (MC) algorithm to estimate without bias the exact \(P\)-value of
this test (Guo and Thompson 1992), and three parameters are needed to
control this algorithm (see Section \ref{algorithms-for-exact-tests}).
These different values may be provided either at Genepop's request, or
through the \texttt{Dememorisation}, \texttt{BatchLength} and
\texttt{BatchNumber} settings. Two results are provided for each test by
the MC algorithm: the estimated \(P\)-value associated with the null
hypothesis of HW equilibrium, and the standard error (S.E.) of this
estimate.

For all tests concerned with sub-options 1-3, there are three possible
cases. The number of distinct alleles at each locus in each sample is

\begin{itemize}
\item
  \textbf{no more than 4:} Genepop will give you the choice between the
  complete enumeration and the MC method. If you have less than 1000
  individuals per sample, the complete enumeration is recommended.
  Otherwise, the MC method could be much faster. But there are no
  general rules, results are highly variable, depending also on allele
  frequencies.
\item
  \textbf{always 5 or more:} Genepop will automatically perform only the
  MC method.
\item
  \textbf{sometimes higher than 4, sometimes not:} For cases where the
  number of alleles is 4 or lower, Genepop will give you the choice
  between both methods. For the other situations (5 alleles or more in
  some samples), the MC method will be automatically performed.
\end{itemize}

Whether one wants enumeration or MC methods to be performed can be
specified at runtime, or otherwise by the \texttt{HWtests}
setting\index{HWtests setting}, with options
\texttt{HWtests=enumeration} and \texttt{HWtests=MCMC}. The default in
the batch mode is \texttt{enumeration}.\index{Hardy-Wein\-berg tests}

\subsection{Output}\label{output}

Results are stored in a file named as follows

\begin{longtable}[]{@{}ll@{}}
\toprule
sub-option & Extension\tabularnewline
\midrule
\endhead
1 & \emph{yourdata}\texttt{.D}\tabularnewline
2 & \emph{yourdata}\texttt{.E}\tabularnewline
3 & \emph{yourdata}\texttt{.P}\tabularnewline
4 & \emph{yourdata}\texttt{.DG}\tabularnewline
5 & \emph{yourdata}\texttt{.EG}\tabularnewline
\bottomrule
\end{longtable}

where \emph{yourdata} is (throughout this document) the name of the
input file.

For each test, several values are indicated on the same line: (i) the
\(P\)-value of the test (or ``-'' if no data were available, or only one
allele was present, or two alleles were detected but one was represented
by only one copy); (ii) the standard error (only if a MC method was
used); (iii) two estimates of \(F_\mathrm{IS}\), Weir and Cockerham
(1984)'s (1984) estimate (W\&C), and Robertson and Hill (1984)'s (1984)
estimate (R\&H). The latter has a lower variance under the null
hypothesis. Finally, the number of ``steps'' is given: for the complete
enumeration algorithm this is the number of different genotypic matrices
considered, and for the Markov chain algorithm the number of
switches\index{Markov chain algorithms!switches} (change of genotypic
matrice) performed.\footnote{New to Genepop 4.0.}

\subsection{Sub-options 4,5: Global tests across loci or across
samples}\label{sub-options-45-global-tests-across-loci-or-across-samples}

For sub-option 3, a global test across loci or across sample is
constructed using Fisher's method.\index{Combination of different tests}
This method (sometimes conservative because discrete probabilities are
analyzed), is only performed for convenience and its relevance should be
first established (e.g.~statistical independence of loci).

General statistical theory shows that there is no uniformly better way
to combine \(P\)-values of different tests. When an alternative model is
specified, it is possible to find a better way of combining results from
different data sets than Fisher's method, and usually not by combining
\(P\)-values. In the present context one such method is the multisample
score test\index{Hardy-Wein\-berg Tests!multisample score test} of
Rousset and Raymond (1995), which defines a global test across loci
and/or across samples generalizing the tests of sub-options 1 and 2. The
global tests are performed by sub-options 4 and 5, only by the MC
algorithm. Independence of loci is also assumed for these global tests.

The output file reports global P value estimates and standard errors per
population, per locus, and over all loci and populations. For each
global P value, the average number of switches per test combined is also
reported. Since it is tempting to reduce the chain length parameters in
this option, special care is needed in checking this accuracy diagnostic
(see p.41).\footnote{Again new to Genepop 4.0.}

This option generates several large temporary files. The space used
temporarily by Genepop can be estimated as: (\#of Loci+\#of
pop+1)*batches*(iterations per batch)*8 octets. For example it will
require about 240 Mo of temporary hard disk space if you have 10 loci,
50 samples and if you use a chain of 500,000 steps (100 batches of 5000
iterations).

\subsection{Analyzing a single genotypic
matrix}\label{analyzing-a-single-genotypic-matrix}

\index{Input format!for single HW test} It is possible to perform a
single HW test independently of the Genepop input file. This option is
not presented in the Genepop menu. You should have an input file with a
genotypic matrix (which can be taken from the output file of option 5
and edited), and use the \texttt{HWfile}\index{HWfile setting}
setting\index{HW program}.\footnote{In earlier versions of Genepop, this
  analysis was done through the HW.BAT batch file.} When Genepop is
launched in this way, the following menu will appear:

\begin{verbatim}
 HW test for each locus in each population:
    H1 = Heterozygote deficiency .................1
    H1 = Heterozygote excess .....................2
    Probability test .............................3

 Allele frequencies, expected genotypes, Fis .... 4
 Quit ........................................... 5
\end{verbatim}

All HW tests corresponding to options 1.1--3 of ``regular'' Genepop are
available through options 1--3, and basic information similar to that
given by regular option 5.1 is available through the present option 4.
Results are stored at the end of your input file. The exact format of
the input file is:

\textbf{First line:} anything. Use this line to store information about
your data.

\textbf{Second line:} The number of alleles \(n\).

\textbf{Line three through} \(n+2\)\textbf{:} the genotypic matrix (see
example).

\textbf{Beyond line} \(n+2\) \textbf{:} anything (this is not read by
the program).

An example with four alleles is:

\begin{verbatim}
 Human Monoamine Oxidase (MOAO) Data
 4
 2
 12 24
 30 34 54
 22 21 20 10
\end{verbatim}

If this file is named \texttt{MOAO}, you can analyze it by setting
\texttt{HWfile=MOAO} in the settings; you can also set
\texttt{HWfileOptions=1}\index{HWfileOptions setting} to run option 1
without making your way through the menus. All this can be done through
the console command line. For example

\begin{verbatim}
Genepop HWFile=MOAO HWfileOptions=1,2,3,4
\end{verbatim}

will perform all four analyses available through the above menu. General
settings \texttt{Dememorisation}, \texttt{BatchLength},
\texttt{BatchNumber}, and \texttt{Mode} all affect these analyses in the
same way as they affect analyses of regular input files.

\subsection{Code checks}\label{code-checks}

\index{Code checks} Code for HW tests has a now venerable history of
testing. Early versions of Genepop were compared with the Exactp step in
Biosys (Swofford and Selander 1989) for two allele cases, and with data
published in Louis and Dempster (1987) and Guo and Thompson (1992) for
more alleles. The sample files \texttt{LouisD87.txt} and
\texttt{GuoT92.txt} contain two such test samples, in single-matrix
format.

\section{Option 2: Tests and tables for linkage
disequilibrium}\label{option-2-tests-and-tables-for-linkage-disequilibrium}

The following menu appears:\footnote{The distinct option 2.3 of Genepop
  3.4 is no longer necessary as option 2.1 of Genepop 4.0 more
  gracefully handles haploid data.}

\begin{verbatim}
Pairwise associations (haploid and genotypic disequilibrium):
      Test for each pair of loci in each population ......... 1
      Only create genotypic contingency tables .............. 2

Menu  ....................................................... 3
\end{verbatim}

\subsection{Sub-option 1: Tests}\label{sub-option-1-tests}

For this option the null hypothesis is: ``Genotypes at one locus are
independent from genotypes at the other locus''. For a pair of diploid
loci, no assumption is made about the gametic phase in double
heterozygotes. In particular, it is not inferred assuming one-locus HW
equilibrium, as such equilibrium is not assumed anywhere in the
formulation of the test. The test is thus one of association between
diploid genotypes at both loci, sometimes described as a test of the
composite linkage disequilibrium\index{Linkage disequilibrium!composite}
(Weir 1996, 126--28). For a haploid locus and a diploid one, a test of
association between the haploid and diploid genotypes is computed (there
is no concern about gametic phase in this case). This makes it easy to
test for cyto-nuclear
disequilibria\index{Linkage disequilibrium!cyto-nuclear}. For a pair of
loci with haploid information, a straightforward test of association of
alleles at the two loci is computed.

The default test statistic is now the log likelihood ratio statistic
(\(G\)-test). However one can still perform probability tests (as
implemented in earlier versions of Genepop) by using the
\texttt{GameticDiseqTest=Proba}\index{GameticDiseqTest setting} setting.

For a given pair of loci within one sample, the relevant information is
represented by a contingency table looking e.g.~like

\begin{verbatim}
       GOT2
       1.1  1.3  3.3  1.7  3.7
EST    _________________________
 1.1   1    1    0    0    1      3
 1.2   16   6    1    3    2      28
       _________________________
       17   7    1    3    3      31
\end{verbatim}

for two diploid loci (\texttt{1.1}, etc., are the diploid genotypes at
each locus). Contingency tables are created for all pairs of loci in
each sample, then a \(G\) test or a probability test for each table is
computed for each table using the Markov chain algorithm of Raymond and
Rousset (1995a). The number of
switches\index{Markov chain algorithms!switches} of the algorithm is
given for each table analyzed.\footnote{This was not the case in earlier
  versions of Genepop}

\subsection{Output}\label{output-1}

Results are stored in the file \emph{yourdata}\texttt{.DIS}. Three
intractable situations are indicated: empty tables (``No data''), table
with one row or one column only (``No contingency table''), and tables
for which all rows or all columns marginal sums are 1 (``No
information''). For each locus pair within each sample, the unbiased
estimate of the P-value is indicated, as well as the standard error.
Next, a global test (Fisher's method) for each pair of loci is performed
across samples.\index{Combination of different tests}

See also the next section for analysis of a single table.

\subsection{Sub-option 2: create
tables}\label{sub-option-2-create-tables}

Suboption 2 only generates the above contingency tables and stores them
in the file \emph{yourdata}\texttt{.TAB}

\subsection{Code checks}\label{code-checks-1}

See code checks for Option 3.

\section{Option 3: population
differentiation}\label{option-3-population-differentiation}

The following menu appears:

\begin{verbatim}
 Testing population differentiation :

      Genic differentiation:
           for all populations ........................ 1
           for all pairs of populations ............... 2

      Genotypic differentiation:
           for all populations ........................ 3
           for all pairs of populations ............... 4

      Main menu  ...................................... 5
\end{verbatim}

All tests are based on Markov chain algorithms. The Markov chain
parameters are controlled exactly as in option 1.

\subsection{Sub-options 1 or 2 (genic
differentiation)}\label{sub-options-1-or-2-genic-differentiation}

\index{Differentiation!genic} They are concerned with the distribution
of alleles is the various samples. The null hypothesis tested is
``alleles are drawn from the same distribution in all populations''. For
each locus, the test is performed on a contingency table like this one:

\begin{verbatim}
          Sub-Pop.  Alleles
                    1    2   Total
                    _______
           1        14   46   60
           2        6    76   82
           3        10   74   84
           4        4    58   62
                    _______
          Total     34   254  288
\end{verbatim}

For each locus, an unbiased estimate of the P-value is computed. The
test statistic is either the probability of the sample conditional on
marginal values, the \(G\) log likelihood ratio, or the level of gene
diversity. In the first case, the test is Fisher's exact probability
test, and the algorithm is described in Raymond and Rousset (1995a). A
simple modification of this algorithm is used for the exact \(G\)
test.\footnote{Up to version 3.4, Genepop only computed Fisher's exact
  test in these sub-options.} Genepop's default is the \(G\) test. You
can revert to Fisher's test by using the
\texttt{DifferentiationTest=Proba}\index{DifferentiationTest setting}
setting. Finally, the level of gene diversity can be used as a test
statistic when coupled with the \texttt{GeneDivRanks} setting (this was
new to version 4.1; see Section
\ref{gene-diversity-as-a-test-statistic}).

For sub-option 2, the tests are the same, but they are performed for all
pairs of samples for all loci.

\subsection{Sub-options 3 or 4 (genotypic
differentiation)}\label{sub-options-3-or-4-genotypic-differentiation}

\index{Differentiation!genotypic} are concerned with the distribution of
diploid genotypes in the various populations. The null hypothesis tested
is ``genotypes are drawn from the same distribution in all
populations''. For each locus, the test is performed on a contingency
table like this one:

\begin{verbatim}
                  Genotypes:
                  -------------------------
                  1    1   2   1   2   3
         Pop:     1    2   2   3   3   3   All
         ----
         Pop1     142  27  0   13  1   0   183
         Pop2     149  20  0   11  0   4   184
         Pop3     131  12  0   9   0   1   153
         Pop4     119  22  1   10  0   0   152
         Pop5     120  17  1   10  1   0   149
         Pop6     134  18  2   15  0   0   169
         Pop7     116  15  1   10  1   1   144
         Pop8     214  41  3   14  2   1   275
         Pop9     84   17  0   7   2   0   110
         Pop10    107  18  0   15  3   0   143
         Pop11    134  32  1   21  4   0   192
         Pop12    105  26  1   11  1   4   148
         Pop13    97   19  2   23  4   0   145
         Pop14    95   28  3   19  3   1   149

         All:     1747 312 15  188 22  12  2296
\end{verbatim}

An unbiased estimate of the P-value of a log-likelihood ratio (\(G\))
based exact test is performed. For this test, the statistics defining
the rejection zone is the \(G\) value computed on the genic table
derived from the genotypic one (see Goudet et al. 1996 for the choice of
this statistic),\index{Differentiation!genic-genotypic test} so that the
rejection zone is defined as the sum of the probabilities of all tables
(with the same marginal genotypic values as the observed one) having a
\(G\) value computed on the derived genic table higher than or equal to
the observed \(G\) value.

For sub-option 4, the test is the same but is performed for all pairs of
samples for all loci.

\subsection{Output}\label{output-2}

For the four sub-options, results are stored in a file named as
follows:\footnote{slightly modified in comparison to earlier versions of
  Genepop}

\begin{longtable}[]{@{}lll@{}}
\toprule
sub-option & test & output file name\tabularnewline
\midrule
\endhead
1 & Probability test & \emph{yourdata}\texttt{.PR}\tabularnewline
1 & \(G\) & \emph{yourdata}\texttt{.GE}\tabularnewline
2 & Probability test & \emph{yourdata}\texttt{.PR2}\tabularnewline
2 & \(G\) & \emph{yourdata}\texttt{.GE2}\tabularnewline
3 & \(G\) & \emph{yourdata}\texttt{.G}\tabularnewline
4 & \(G\) & \emph{yourdata}\texttt{.2G2}\tabularnewline
\bottomrule
\end{longtable}

All contingency tables are saved in the output file. Two intractable
situations are indicated: empty tables or tables with one row or one
column only (``No table''), and tables for which all rows or all columns
marginal sums are 1 (``No information''). Estimates of P-values are
given, as well as (for sub-options 1 and 3) a combination of all test
results (Fisher's method), which assumes a statistical independence
across loci. For sub-options 2 and 4, this combination of all tests
across loci (Fisher's method) is performed for each sample pair. The
result \texttt{Highly\ sign.}{[}ificant{]} is reported when at least one
of the individual tests being combined yielded a zero \(P\)-value
estimate.

\subsection{Gene diversity as a test
statistic}\label{gene-diversity-as-a-test-statistic}

\index{Differentiation!gene diversity}

\begin{verbatim}
 DifferentiationTest=GeneDiv
 GeneDivRanks=2,1,3,3,3
\end{verbatim}

\index{DifferentiationTest setting} \index{GeneDivRanks setting}
\texttt{DifferentiationTest=GeneDiv} makes Genepop use gene diversity as
test statistic in tests of genetic differentiation (option 3). The test
will look for a decrease in gene diversity from populations ranked first
(value \texttt{1} in \texttt{GeneDivRanks}) to populations ranked last.
This should work for both genic and genotypic tables, and for pairwise
comparisons as well as for all populations, i.e.~for all sub-options 3.1
to 3.4. The test statistic is
\[\sum_{\textrm{all subsamples $i$}}\sum_{j>i} (Q_j-Q_i)(R_j-R_i)\]
where \(Q_i\) is gene identity in subsample \(i\) and \(R_i\) is the
\texttt{GeneDivRanks} value for this subsample.

\index{Haploid data}

This option also works on input files in contingency table format
(\texttt{strucfile} setting). In that case each \emph{row} of the table
is interpreted as a new population.

\subsection{Analyzing a single contingency
table}\label{analyzing-a-single-contingency-table}

\index{Input format!for single contingency table} It is possible to
analyse any contingency table independently of the Genepop input file.
You should have an input file with a contingency table, and use the
\texttt{strucFile}\index{StrucFile setting}
setting\index{Struc program}.\footnote{In previous versions of Genepop,
  this analysis was done by the Struc program called through the
  \texttt{Struc.BAT} batch file.} This option is not presented in the
Genepop menu. Both the \(G\) and probability tests are available and
performed as in option 3.1. Results are stored at the end of your input
file. An example of input file is:

\begin{verbatim}
 Dull example
 6 5
 1   2  5 10 11
 2   0  8 11 15
 0   0  1  5  6
 10 15 20 51 55
 0   0  0  2  1
 4   5  6 11 10
\end{verbatim}

If this file is named \texttt{structest}, you can analyze it by writing
\texttt{StrucFile=structest} in the settings file, or by the console
command line

\begin{verbatim}
Genepop StrucFile=structest
\end{verbatim}

The exact format of the input file is:

\textbf{First line:} anything. Use this line to store information about
your data.

\textbf{Second line:} The numbers of rows (\(n\)) and columns.

\textbf{Line three through} \(n+2\) \textbf{:} the contingency table
(see example).

\textbf{Beyond line} \(n+2\) \textbf{:} anything (this is not read by
the program).

The default is to perform a \(G\) test, but as in options 3.1 and 3.2
you can revert to Fisher's exact test by the setting
\texttt{DifferentiationTest=Proba}.

\subsection{Code checks}\label{code-checks-2}

\index{Code checks} Code for contingency tables also has a venerable
history of testing. Early versions of Genepop were tested by comparison
with published data (e.g. Mehta and Patel 1983) or by hand calculations.
The example file \texttt{MehtaP83.txt} contains one such test sample.

\section{Option 4: private alleles}\label{option-4-private-alleles}

This option provides a multilocus estimate of the effective number of
migrants (\(Nm\))by Barton and Slatkin's (1986) method. Three estimates
of \(Nm\) are provided, using the three regression lines published in
that reference, and a corrected estimate is provided using the values
from the closest regression line. Results are stored in the file
\emph{yourdata}\texttt{.PRI}.

\section{\texorpdfstring{Option 5: Basic information, \(F_\mathrm{IS}\),
and gene
diversities}{Option 5: Basic information, F\_\textbackslash{}mathrm\{IS\}, and gene diversities}}\label{option-5-basic-information-f_mathrmis-and-gene-diversities}

The following menu appears:

\begin{verbatim}
      Allele and genotype frequencies per locus and per sample .. 1

      Gene diversities & Fis :
                                  Using allele identity ......... 2
                                  Using allele size ............. 3

      Main menu ................................................. 4
\end{verbatim}

\subsection{Sub-option 1: Allele and genotype
frequencies}\label{sub-option-1-allele-and-genotype-frequencies}

This option provides basic information on the data set. The output file
is saved in the file \emph{yourdata}\texttt{.INF}. For each locus in
each sample, several variables are calculated:

\begin{itemize}
\item
  allele frequencies.
\item
  observed and expected genotype proportions.
\item
  \(F_\mathrm{IS}\) estimates for each allele following Weir and
  Cockerham (1984).
\item
  global estimate of \(F_\mathrm{IS}\) over alleles according to Weir
  and Cockerham (1984) (W\&C) and Robertson and Hill (1984) (R\&H).
\item
  observed and ``expected'' numbers of homozygotes and heterozygotes.
  ``Expected'' here means the expected numbers, conditional on observed
  allelic counts, under HW equilibrium; the difference from naive
  products of observed allele frequencies is sometimes called Levene's
  correction, after\index{Levene's correction} Levene (1949).
\item
  the genotypic matrix.
\end{itemize}

A table of allele frequencies for each locus and for each sample is also
computed.

\subsection{\texorpdfstring{Sub-option 2: Identity-based gene
diversities and
\(F_\mathrm{IS}\)}{Sub-option 2: Identity-based gene diversities and F\_\textbackslash{}mathrm\{IS\}}}\label{sub-option-2-identity-based-gene-diversities-and-f_mathrmis}

This option takes the observed frequencies of identical pairs of genes
as estimates (\(Q\)) of corresponding probabilities of identity (\(Q\))
and then simply computes diversities as \(1-Q\): gene diversity within
individuals (\texttt{1-Qintra}), and among individuals within samples
(\texttt{1-Qinter}), per locus per sample, and averaged over samples or
over loci. One-locus \(F_\mathrm{IS}\) estimates are also computed in a
way consistent with Weir and Cockerham (1984). No estimate is given when
no information is available (e.g.~no estimate of diversity between
individuals within a sample when only one individual has been
genotyped).

For haploid data\index{Haploid data}, only the gene diversity among
individuals is computed. Multilocus estimates ignore haploid loci, or on
the contrary ignore diploid loci if the setting
\texttt{EstimationPloidy=Haploid} is used. Single-locus estimates are
computed for both haploid and diploid loci irrespective of this setting.

The output is saved in the file \emph{yourdata}\texttt{.DIV}.

\subsection{\texorpdfstring{Sub-option 3: Allele size-based gene
diversities and
\(\rho_{\mathrm{IS}}\)}{Sub-option 3: Allele size-based gene diversities and \textbackslash{}rho\_\{\textbackslash{}mathrm\{IS\}\}}}\label{sub-option-3-allele-size-based-gene-diversities-and-rho_mathrmis}

Option 5.3 is analogous to option 5.2. It computes measures of diversity
based on allele size, namely mean squared allele size differences within
individuals (\texttt{MSDintra}), and among individuals within samples
(\texttt{MSDinter}), per locus per sample, and averaged over samples or
over loci. Corresponding \(\rho_\mathrm{IS}\) (the \(F_\mathrm{IS}\)
analogue, see Section \ref{rho-stats}) estimates are also computed.
Allele size is the allele name unless it has been given through the
\texttt{AlleleSizes} setting.

For haploid data, only the mean squared difference \texttt{MSDinter}
among individuals is computed. Multilocus estimates ignore haploid loci,
or on the contrary ignore diploid loci if the setting
\texttt{EstimationPloidy=Haploid} is used. Single-locus estimates are
computed for both haploid and diploid loci irrespective of this setting.

The output is saved in the file \emph{yourdata}\texttt{.MSD}.

\section{Option 6: Fst and other correlations, isolation by
distance}\label{option-6-fst-and-other-correlations-isolation-by-distance}

The following menu appears:

\begin{verbatim}
 Estimating spatial structure:

 The information considered is :
      --> Allele identity (F-statistics)
                For all populations ............ 1
                For all population pairs ....... 2
      --> Allele size (Rho-statistics)
                For all populations ............ 3
                For all population pairs ....... 4

 Isolation by distance
                between individuals ............ 5
                between groups.................. 6

    Main menu  ................................. 7
\end{verbatim}

\begin{longtable}[]{@{}llll@{}}
\caption{\label{tab:isolstats} Genetic distance statistics available in
options 6.5 and 6.6}\tabularnewline
\toprule
\begin{minipage}[b]{0.08\columnwidth}\raggedright\strut
Data ploidy\strut
\end{minipage} & \begin{minipage}[b]{0.16\columnwidth}\raggedright\strut
pop = individual?\strut
\end{minipage} & \begin{minipage}[b]{0.19\columnwidth}\raggedright\strut
\texttt{isolationStatistic} setting\strut
\end{minipage} & \begin{minipage}[b]{0.45\columnwidth}\raggedright\strut
Estimator used\strut
\end{minipage}\tabularnewline
\midrule
\endfirsthead
\toprule
\begin{minipage}[b]{0.08\columnwidth}\raggedright\strut
Data ploidy\strut
\end{minipage} & \begin{minipage}[b]{0.16\columnwidth}\raggedright\strut
pop = individual?\strut
\end{minipage} & \begin{minipage}[b]{0.19\columnwidth}\raggedright\strut
\texttt{isolationStatistic} setting\strut
\end{minipage} & \begin{minipage}[b]{0.45\columnwidth}\raggedright\strut
Estimator used\strut
\end{minipage}\tabularnewline
\midrule
\endhead
\begin{minipage}[t]{0.08\columnwidth}\raggedright\strut
Diploid\strut
\end{minipage} & \begin{minipage}[t]{0.16\columnwidth}\raggedright\strut
Yes (option 6.5)\strut
\end{minipage} & \begin{minipage}[t]{0.19\columnwidth}\raggedright\strut
\texttt{=a}\strut
\end{minipage} & \begin{minipage}[t]{0.45\columnwidth}\raggedright\strut
\(\hat{a}\)\strut
\end{minipage}\tabularnewline
\begin{minipage}[t]{0.08\columnwidth}\raggedright\strut
Diploid\strut
\end{minipage} & \begin{minipage}[t]{0.16\columnwidth}\raggedright\strut
Yes (option 6.5)\strut
\end{minipage} & \begin{minipage}[t]{0.19\columnwidth}\raggedright\strut
\texttt{=e}\strut
\end{minipage} & \begin{minipage}[t]{0.45\columnwidth}\raggedright\strut
\(\hat{e}\)\strut
\end{minipage}\tabularnewline
\begin{minipage}[t]{0.08\columnwidth}\raggedright\strut
Diploid\strut
\end{minipage} & \begin{minipage}[t]{0.16\columnwidth}\raggedright\strut
No (option 6.6)\strut
\end{minipage} & \begin{minipage}[t]{0.19\columnwidth}\raggedright\strut
none (default)\strut
\end{minipage} & \begin{minipage}[t]{0.45\columnwidth}\raggedright\strut
\(F_\mathrm{ST}\)/(1-\(F_\mathrm{ST}\))\strut
\end{minipage}\tabularnewline
\begin{minipage}[t]{0.08\columnwidth}\raggedright\strut
Diploid\strut
\end{minipage} & \begin{minipage}[t]{0.16\columnwidth}\raggedright\strut
No (option 6.6)\strut
\end{minipage} & \begin{minipage}[t]{0.19\columnwidth}\raggedright\strut
\texttt{=singleGeneDiv}\strut
\end{minipage} & \begin{minipage}[t]{0.45\columnwidth}\raggedright\strut
\(F/(1-F)\) variant with denominator common to all pairs\strut
\end{minipage}\tabularnewline
\begin{minipage}[t]{0.08\columnwidth}\raggedright\strut
Haploid\strut
\end{minipage} & \begin{minipage}[t]{0.16\columnwidth}\raggedright\strut
Yes (option 6.5)\strut
\end{minipage} & \begin{minipage}[t]{0.19\columnwidth}\raggedright\strut
none (default)\strut
\end{minipage} & \begin{minipage}[t]{0.45\columnwidth}\raggedright\strut
\(\hat{a}\)-like statistic with stand-in for within-deme gene
diversity\strut
\end{minipage}\tabularnewline
\begin{minipage}[t]{0.08\columnwidth}\raggedright\strut
Haploid\strut
\end{minipage} & \begin{minipage}[t]{0.16\columnwidth}\raggedright\strut
No (option 6.6)\strut
\end{minipage} & \begin{minipage}[t]{0.19\columnwidth}\raggedright\strut
none (default)\strut
\end{minipage} & \begin{minipage}[t]{0.45\columnwidth}\raggedright\strut
\(F_\mathrm{ST}\)/(1-\(F_\mathrm{ST}\))\strut
\end{minipage}\tabularnewline
\begin{minipage}[t]{0.08\columnwidth}\raggedright\strut
Haploid\strut
\end{minipage} & \begin{minipage}[t]{0.16\columnwidth}\raggedright\strut
No (option 6.6)\strut
\end{minipage} & \begin{minipage}[t]{0.19\columnwidth}\raggedright\strut
\texttt{=singleGeneDiv}\strut
\end{minipage} & \begin{minipage}[t]{0.45\columnwidth}\raggedright\strut
\(F/(1-F)\) variant with denominator common to all pairs\strut
\end{minipage}\tabularnewline
\bottomrule
\end{longtable}

Suboptions 5 and 6 provide a variety of analyses of isolation by
distance patterns, including bootstrap confidence intervals of the slope
of spatial regression (or equivalently, for ``neighborhood'' size
estimates). Starting with version 4.1, it is even possible to test given
values of the slope, through the \texttt{testPoint} setting; and
additional estimators (merely minor variation on a common logic) have
been implemented, in particular for haploid data. Table
\ref{tab:isolstats} summarizes the choice of methods, each of which will
now be detailed.

\subsection{\texorpdfstring{Sub-options 1--4: \(F\)-statistics and
\(\rho\)-statistics}{Sub-options 1--4: F-statistics and \textbackslash{}rho-statistics}}\label{sub-options-14-f-statistics-and-rho-statistics}

These options compute estimates of \(F_\mathrm{IS}\), \(F_\mathrm{IT}\)
and \(F_\mathrm{ST}\) or analogous correlations for allele size, either
for each pair of population (sub-options 2 and 4) or a single measure
for all populations (sub-options 1 and 3). \(F_\mathrm{ST}\) is
estimated by a ``weighted'' analysis of variance Cockerham (1973; Weir
and Cockerham 1984), and the analogous measure of correlation in allele
size (\(\rho_\mathrm{ST}\)) is estimated by the same technique (see
Section \ref{rho-stats}). Multilocus estimates are computed as detailed
in Section \ref{Fmulti}). For haploid data,\index{Haploid data} remember
to use the \texttt{EstimationPloidy=Haploid} setting.

In sub-option 1, the output is saved in the file
\emph{yourdata}\texttt{.FST}. Beyond \(F_\mathrm{IS}\),
\(F_\mathrm{IT}\) and \(F_\mathrm{ST}\) estimates, estimation of
within-individual gene diversity and within-population among-individual
gene diversity are reported as in option 5.2.

In sub-option 2 (pairs of populations), single locus and multilocus
estimates are written in the \emph{yourdata}\texttt{.ST2} file and
multilocus estimates are also written in the
\emph{yourdata}\texttt{.MIG} file in a format suitable for analysis of
isolation by distance (see option 6.6 for further details).

Sub-option 3 is analogous to sub-option 1, but for allele-size based
estimates. the output is saved in the file \emph{yourdata}\texttt{.RHO}.
Beyond \(\rho_\mathrm{IS}\), \(\rho_\mathrm{IT}\) and
\(\rho_\mathrm{ST}\) estimates, estimation of within-individual gene
diversity and within-population among-individual gene diversity are
reported as in option 5.3.

Sub-option 4 is analogous to sub-option 2, but for allele-size based
estimates. Output file names are as in sub-option 2.

\subsection{Sub-option 5: isolation by distance between
individuals}\label{sub-option-5-isolation-by-distance-between-individuals}

This option allows analysis of isolation by distance between pairs of
individuals. It provides estimates of ``neighborhood
size'',\index{Neighborhood size|see{$D\sigma^2$ estimation}} or more
precisely of \(D\sigma^2\), the product of population density and axial
mean square parent-offspring distance, derived from the slope of the
regression of pairwise genetic statistics against geographical distance
or log(distance) in linear or two-dimensional habitats, respectively.
More details are described in Rousset (2000) (\(\hat{a}\) statistic),
Leblois, Estoup, and Rousset (2003) (bootstrap confidence intervals) and
Watts et al. (2007) (\(\hat{e}\) statistic). For haploid data, a proxy
for the \(\hat{a}\) statistic has been introduced in version 4.1.

The position of individuals must be specified as two coordinates
standing for their name (i.e.~before the comma on the line for each
individual), and since each individual is considered as a sample, it
must be separated by a \texttt{Pop}. An example of such input file is
given below: The first individual is located at the point \(x = 0.0\),
\(y = 15.0\) (showing that the decimal separator is a period), the
second at the point \(x = 0\), \(y =30\), etc. This example also shows
that \emph{individual identifiers can be added after these coordinates}.

\begin{verbatim}
 Title line: A really too small data set
 ADH Locus 1
 ADH #2
 ADH three
 ADH-4
 ADH-5
 Pop
 0.0 15.0,  0201 0303 0102 0302 1011
 Pop
 0 30 Second indiv,  0202 0301 0102 0303 1111
 Pop
 0 45,  0102 0401 0202 0102 1010
 Pop
 0 60,  0103 0202 0101 0202 1011
 Pop
 0 75,  0203 0204 0101 0102 1010
 POP
 15 15,      0102 0202 0201 0405 0807
 Pop
 15 30,      0102 0201 0201 0405 0307
 Pop
 15 45,      0201 0203 0101 0505 0402
 Pop
 15 60,      0201 0303 0301 0303 0603
 Pop
 15 75,      0101 0201 0301 0505 0807
\end{verbatim}

\textbf{Missing information} arises when there is no genetic estimate
(if a pair of individuals has no genotypes for the same locus, for
example), or when geographic distance is zero and log(distance) is used.
Genepop will correctly handle such missing information until it comes to
the point where regression cannot be computed or there are not several
loci to bootstrap over.

Options to be described within option 6.5 are: \(\hat{a}\) or
\(\hat{e}\) pairwise statistics (for diploid data); log transformation
for geographic distances; minimal geographic distance; coverage
probability of confidence interval; testing a given value of the slope;
Mantel test settings; conversion to genetic distance matrix in Phylip
format. Allele-size based analogues of \(\hat{a}\) or \(\hat{e}\) can be
defined, but they should perform very poorly (Leblois, Estoup, and
Rousset 2003; Rousset 2007), so such an analysis has been purposely
disabled.

\textbf{Pairwise statistics for diploid data}: They are selected by the
setting
\texttt{IsolationStatistic=a}\index{IsolationStatistic setting}\index{Dsigma2@$D\sigma^2$
estimation!a@$\hat{a}$ statistic} or \texttt{=e}, or at runtime (in
batch mode, the default is \(\hat{a}\)). The \(\hat{e}\) statistic is
asymptotically biased in contrast to \(\hat{a}\), but has lower
variance. The bias of the \(\hat{e}\)-based slope is higher the more
limited dispersal is, so it performs less well in the lower range of
observed dispersal among various species. Confidence intervals are also
biased (Leblois, Estoup, and Rousset 2003; Watts et al. 2007), being too
short in the direction of low \(D\sigma^2\) values, and on the contrary
conservative in the direction of low \(D\sigma^2\) values. Based on the
simulation results of Watts et al. (2007), a provisional advice is to
run analyses with both statistics, and to derive an upper bound for the
\(D\sigma^2\) confidence interval (CI), hence the lower bound for the
regression slope, from \(\hat{e}\) (which has CI shorter than
\(\hat{a}\), though still conservative) and the other \(D\sigma^2\)
bound, hence the upper bound for the regression slope, from \(\hat{a}\)
(which has too short CI, but less biased than the \(\hat{e}\) CI). When
the \(\hat{e}\)-based \(D\sigma^2\) estimate is below 2500 (linear
habitat) or 4 (two-dimensional habitat) it is suggested to derive both
bounds from \(\hat{a}\).

Note that \(\hat{e}\) is essentially Loiselle's
statistic\index{Dsigma2@$D\sigma^2$ estimation!Loiselle's statistic}
(Loiselle et al. 1995), which use in this context has previously been
advocated by e.g. Vekemans and Hardy (2004).

For \textbf{haploid data} (i.e. \texttt{EstimationPloidy=Haploid}) the
denominators of the \(\hat{a}\) and \(\hat{e}\) statistics cannot be
computed. Ideally the denominator should be the gene diversity among
individuals that would compete for the same position, as could be
estimated from ``group'' data. As a reasonable first substitute, Genepop
uses a single estimate of gene diversity (from the total sample and for
each locus) to compute the denominators for all pairs of individuals.
This amount to assume that overall differentiation in the population is
weak.

\textbf{Log transformation for geographic distances}: This
transformation is required for estimation of \(D\sigma^2\) when
dispersal occurs over a surface rather than over a linear habitat. It is
the default option in batch mode. It can be turned on and off by the
setting \texttt{GeographicScale=Log}\index{GeographicScale setting} or
\texttt{=Linear} or equivalently by \texttt{Geometry=2D} or
\texttt{=1D}.\index{Geometry setting}

\textbf{Coverage probability of confidence interval} This is the target
probability that the confidence interval contains the parameter value.
The usage is to compute intervals with 95\% coverage and equal 2.5\%
tails, and this is the default coverage in Genepop. This can be changed
by the setting \texttt{CIcoverage}, e.g. \texttt{CIcoverage=0.99} will
compute interval with target probabilities 0.5\% that either the
confidence interval is too low or too high (an unrealistically large
number of loci may be necessary to achieve the latter
precision).\index{CIcoverage setting}\index{Confidence intervals}

\textbf{Minimal and maximal geographic distances:} As discussed in
Rousset (1997), samples at small geographic distances are not expected
to follow the simple theory of the regression method, so the program
asks for a minimum geographical distance. Only pairwise comparisons of
samples at larger distances are used to estimate the regression
coefficient (all pairs are used for the Mantel test). The minimal
distance may be specified by the setting
\texttt{MinimalDistance=}\emph{value}\index{MinimalDistance setting} or
at runtime. This being said, it is wise to include all pairs in the
estimation as no substantial bias is expected, and this avoids
uncontrolled hacking of the data. Thus, the suggested minimal distance
here is any distance large enough to exclude only pairs at zero
geographical distance. Only non-negative values are accepted, and the
default in batch mode is 0.0001.

There is also a setting
\texttt{MaximalDistance=}\emph{value}.\index{MaximalDistance setting}
This should not be abused, and is (therefore) available only through the
settings file, not as a runtime option.

\textbf{Testing a given value of the slope} The setting
\texttt{testPoint=0.00123} (say) returns the unidirectional P-value for
a specific value of the slope, using the ABC bootstrap method. This is
the reciprocal of a confidence interval computation: confidence
intervals evaluate parameter values corresponding to given error levels,
say the 0.025 and 0.975 unidirectional levels for a 95\% bidirectional
CI, while this option evaluates the unidirectional P-value associated
with a given parameter value.

\textbf{Mantel test:}\index{Mantel test} The Mantel test is implemented.
See Section \ref{mantel-test} for limitations of this test. In the
present context this is an exact test of the null hypothesis that there
in no spatial correlation between genetic samples.

Up to version 4.3 Genepop implemented only a Mantel test based on the
rank correlation. It now also implements, and performs by default,
Mantel tests based on the regression coefficient for the ``genetic
distance'' statistic used to quantify isolation by distance. The latter
tests should generally be more congruent with the confidence intervals
based on the same distances than the rank-based tests are. The rank test
can now be performed by using the setting \texttt{MantelRankTest=} (no
\emph{value} needed).\index{MantelRankTest setting}

Ideally the confidence interval for the slope should contain zero if and
only if the Mantel test is non-significant. Some exceptions may occur as
the bootstrap method is only approximate, but such exceptions appear to
be rare. Exceptions may more commonly occur when the bootstrap is based
on the regression of genetic ``distance'' and geographic distance over a
selected range of the latter.

The number of permutations may be specified by the setting
\texttt{MantelPermutations=}\emph{value},\index{MantelPermutations setting}
or else at runtime. In batch mode, if no such value has been given the
default behaviour is not to perform the test.

\textbf{Export genetic distance matrix in Phylip
format}.\index{Phylip package|see PhylipMatrix} This option is activated
by the setting \texttt{PhylipMatrix=} (no \emph{value}
needed).\index{PhylipMatrix setting} It may be useful, if you wish to
use Phylip, to draw a tree based on genetic distances. A constant is
added to all values if necessary so that all resulting distances are
positive. Output is written in the file \emph{yourdata}\texttt{.PMA}. No
further estimation or testing is done, so the name of the
groups/individuals does not need to be their spatial coordinates.

Except for this export option, output files are:

\begin{itemize}
\item
  the \emph{yourdata}\texttt{.ISO} output file, containing (i) a genetic
  distance (\(\hat{a}\) or \(\hat{e}\)) half-matrix and a geographic
  (log-)distance half-matrix; missing information is reported as
  `\texttt{-}'; (ii) regression estimates and bootstrap confidence
  intervals; (iii) the result of testing a slope value (using
  \texttt{testPoint}); (iv) results of a Mantel test for evidence of
  isolation by distance, if requested; (v) a bootstrap interval for the
  intercept. The order of elements in the half-matrices is:

\begin{verbatim}
       1     2     3
 2     x
 3     x     x
 4     x     x     x
\end{verbatim}
\item
  a \emph{yourdata}\texttt{.MIG} output file, containing the same
  genetic and geographic distances as in the \texttt{ISO} file, but with
  more digits, and without estimation or test results. This file was
  formerly useful as input for the Isolde program (see ``Former option 5
  of Genepop'', below), and is a bit redundant now.
\item
  a \emph{yourdata}\texttt{.GRA} output file, where again the genetic
  and geographic distances are reported, now as \((x,y)\) coordinates
  for each pair of individuals (one per line). This is useful e.g.~for
  importing the output into programs with good graphics. Pairs with
  missing values (either \(x\) or \(y\)) are not reported in this file.
\end{itemize}

\subsection{Sub-option 6: isolation by distance between
groups}\label{sub-option-6-isolation-by-distance-between-groups}

This option is analogous to the previous one, but derives \(D\sigma^2\)
estimates from a regression of
\(F_{\mathrm{ST}}\)/(1-\(F_{\mathrm{ST}}\))*
estimates\index{Dsigma2@$D\sigma^2$ estimation!Fst@$\Fst/(1-\Fst)$ statistic}
to geographic distance in a linear habitat, or log(distance) in a
two-dimensional habitat (Rousset 1997).

Both diploid and haploid data (through
\texttt{EstimationPloidy=Haploid}) are handled. Missing information is
handled as in option 6.5. Input format is the same, except that some
samples must contain several individuals. The coordinates of each sample
are still contained in the name of each sample, that is in the name of
the last individual in each sample.

In addition some allele-size based analyses are possible (by the setting
\texttt{AllelicDistance=Size}) but again they are not advised in
general. Further options within option 6.6 are:
\texttt{isolationStatistic}; \texttt{SingleGeneDiv}; minimal geographic
distance; log transformation for geographic distances; testing a given
value of the slope; Mantel test settings; conversion to genetic distance
matrix in Phylip format. They operate as described above for analyses
between individuals, the only difference being the genetic distance used
(see Table \ref{tab:isolstats}). In particular, a minor variant of the
\(F/(1-F)\) estimator is introduced in version 4.1, by analogy to the
``between individuals'' estimators. Recall that
\(F/(1-F)=(Q_r-Q_0)/(1-Q_0)\) where \(1-Q_0\) is the within-deme gene
diversity. The \(F/(1-F)\) method uses per-pair estimates of this
within-deme gene diversity, which may not be best. With
\texttt{IsolationStatistic=SingleGeneDiv} a single estimate is used for
all pairwise statistics. In principe this should be better when small
per-group samples are considered, but the generic \(F/(1-F)\) method is
still available as the default method. Limited testing so far suggests
little effect of the choice of the statistic on inferences from samples
with 10 haploid individuals per group and high overall diversity.

Output is written in three files \emph{yourdata}\texttt{.ISO},
\emph{yourdata}\texttt{.MIG}, and \emph{yourdata}\texttt{.GRA} with the
same contents as in option 6.5, except for the nature of the genetic
distances.

\subsection{Former sub-option 5 of Genepop: analysis of isolation by
distance from a genetic distance
matrix}\label{former-sub-option-5-of-genepop-analysis-of-isolation-by-distance-from-a-genetic-distance-matrix}

That option (using the Isolde program)\index{Isolde program} allowed one
to perform the analyses of sub-options 5 and 6 from a file with two
semi-matrices, one for genetic ``distances'' \(F_{\mathrm{ST}}\) or
whatever), the other for Euclidian distances. These analyses are now
available through the
\texttt{IsolationFile}\index{IsolationFile setting} setting. Most
choices within options 6.5 and 6.6 are available through this option,
and missing data are handled\footnote{more extensively than in earlier
  versions of Genepop.} (see example below). However, it is not possible
to compute nonparametric confidence intervals for the regression slope
since per-locus information is not provided (remarkably, some software
pretends to compute nonparametric intervals in this case). This option
may serve as a general purpose program for Mantel tests. Of course, some
settings (minimal geographic distance, the \(F/(1-F)\) transformation,
and the interpretation of one one-tailed \(P\) value as a test of
isolation by distance) make sense in the narrower inference context of
options 6.5 and 6.6.

The option is called by \texttt{IsolationFile=}\emph{input file name}
where the input file follows the
format\index{Input format!for Mantel test} of the
\emph{yourdata}\texttt{.MIG} file written by options 6.5 and 6.6, which
may be used as models. An example is

\begin{verbatim}
 Lousy data                   <------anything (comments)
 8 (an example)                      <---# of samples (comments ignored)
 Fst estimates:                              <---anything (comments)
  0.003
  0.18 0.107
  0.19 0.068  0.011
  0.20 0.664  0.665 0.009
  0.21 0.098    -   0.673  0.675
  0.22 0.048  0.682  0.683  0.017  0.001
  0.23 0.715  0.721  0.666  0.666  0.037 0.006
 distances:                          <---anything (comments)
  158.0
  158.0 1215.0
  158.1 1213.0 2300.0
  158.2 2300.0    2.0 1057.0
  158.3 1055.0 2525.0 2525.0 1000.0
  158.4 1057.0 1055.0 2525.0 2525.0 1000.0
   - 3582.0 3582.0 3582.0 3582.0    1.0 2.222
 Anything after the second half matrix       <----as it says
 is ignored
\end{verbatim}

The order of elements in the half-matrices is again

\begin{verbatim}
       1     2      3
 2     x
 3     x     x
 4     x     x     x
\end{verbatim}

Again as in options 6.5 and 6.6, both missing genetic and geographic
information (`\texttt{-}') are handled.

Output is written at the end of the input file, and as in options 6.5
and 6.6, \((x,y)\) data points are also written in the file
\emph{yourdata}\texttt{.GRA}.

\texttt{Genepop\ IsolationFile=}\emph{input file name}
\texttt{MantelRankTest=} will further replicate the rank test of the old
Isolde program.

\subsection{User-provided geographic distance
matrices}\label{user-provided-geographic-distance-matrices}

The setting \texttt{geoDistFile=}\emph{file
name}\index{geoDistFile setting}\footnote{New to Genepop 4.2} can be
used to provide a geographic distance matrix. Its format is that of
other geographic distances matrices, with one required line of comment:

\begin{verbatim}
 Geographic distances:                 <---anything (comments)
  21
  31 32
  41 42 43
  ...
\end{verbatim}

The number of samples does not need to be given.

\subsection{Analysis of isolation by distance from multiple genetic
distance
matrices}\label{analysis-of-isolation-by-distance-from-multiple-genetic-distance-matrices}

If another program has generated \(F_{\mathrm{ST}}\) or
\(F_{\mathrm{ST}}\)/(1 - \(F_{\mathrm{ST}}\)) matrices for a number of
loci, the computation of bootstrap confidence intervals is possible.
Analysis of such data sets is allowed by the
\texttt{MultiMigFile=}\emph{input file name}
setting.\index{MultiMigFile setting} The format of the input file is the
same as for a single genetic matrix, except that it contains multiple
matrices and that the number of genetic matrices must be given (third
line of input):

\begin{verbatim}
 More lousy data
 8
 16 loci (for example)                 <---# of samples (comments ignored)
 locus 1:                              <---anything (comments)
...                                    <-half matrix (not shown here)
 locus 2:                              <---anything (comments)
...
...                                    <-more loci and half matrices (not shown here)
...
 locus 16:                             <---anything (comments)
...
 Geographic distances:                 <---anything (comments)
  158.0
  158.0 1215.0
  158.1 1213.0 2300.0
  158.2 2300.0    2.0 1057.0
  158.3 1055.0 2525.0 2525.0 1000.0
  158.4 1057.0 1055.0 2525.0 2525.0 1000.0
   - 3582.0 3582.0 3582.0 3582.0    1.0 2.222
 Anything after the second half matrix       <----as it says
 is ignored
\end{verbatim}

The main use of this option is to allow analyses based on genetic
distances not considered in Genepop. If the same estimates are input as
would be computed by Genepop, the results should be similar to those
from options 6.5 and 6.6, but not identical in general, because
Genepop's bootstrap estimates are computed as ratio of weighted average
numerators and denominators of genetic estimates, while
\texttt{MultiMigFile} can only use weighted averages of the ratios,
i.e.~of the input genetic values.

\subsection{Analysis of mean
differentiation}\label{analysis-of-mean-differentiation}

\index{MeanDifferentiationTest setting} It is possible to perform a
bootstrap analysis of the mean pairwise differentiation, through all
menu options that lead to bootstrap analyses of isolation by distance,
when additionally using the setting
\texttt{MeanDifferentiationTest=TRUE}. It takes into account selection
of data by both \texttt{PopTypes} and range of geographical distances.

\section{Data selection for analyses of isolation by
distance}\label{data-selection-for-analyses-of-isolation-by-distance}

\subsection{Selecting a subset of
samples}\label{selecting-a-subset-of-samples}

\index{PopTypes setting} \index{PopTypeSelection setting} The settings
\texttt{PopTypes} and \texttt{PopTypeSelection} have been developed to
facilitate comparison of differentiation patterns within and among
different ecotypes or host races.\index{Population type selection} They
are used as follows:

\begin{verbatim}
 PopTypes= 1 1 2 1 2 1 1 2 3 4
 PopTypeSelection=only 1
 // PopTypeSelection=inter 1 2
 // PopTypeSelection=all
\end{verbatim}

\texttt{PopTypes} allows to distinguish different types of samples
(e.g.~different ecotypes) by integer indices. The number of indices must
match the number of samples in the data file.

\texttt{PopTypeSelection} allows performing analyses (genetic distance
regressions, confidence intervals, Mantel tests) only on pairs of
populations belonging to the types specified. That is, the genetic
differentiation statistic among excluded pairs is not used in any of
these analyses. The different choices are shown above: \texttt{all}
excludes no pairs (this is the default value); \texttt{inter} \(a\)
\(b\) will exclude all pairs that do not involve both types \(a\) and
\(b\) (only two types can be specified); and \texttt{only} \(a\) will
exclude all pairs that involve a type different from \(a\) (only one
type can be specified). For the latter two choices, permutations are
made only among samples from a given type. \texttt{inter\_all\_types}
excludes all pairs within types; no Mantel test is performed in that
case. \texttt{intra\_all\_types} keeps all pairs within types, and
performs a single regression for all types; again, no Mantel test is
performed in that case.

You have to perform the ``\texttt{only}'' and ``\texttt{inter}''
analyses in distinct Genepop runs if you wish to compare their results.
Rousset (1999) explains how inferences can be made from such
comparisons. Note that in this perspective, some comparison of the
intercept may be useful and that Genepop also provides confidence
intervals on the intercept at zero distance {[}or log(distance){]}.

\emph{The inter-type Mantel test may be
misleading}.\index{Mantel test!intertype} The null hypothesis implied by
the permutation procedure is that there is no isolation by distance
among populations within each type, rather than the often more relevant
hypothesis that spatial processes within each type of populations are
independent from each other. For this reason, a more appropriate test of
the latter hypothesis is whether the bootstrap confidence interval for
the inter-types regression slope includes zero or not.

\section{Option 7: File conversions}\label{option-7-file-conversions}

This option allows the conversion of the Genepop input file toward other
formats required by some other programs (the ``ecumenical'' function of
Genepop). Given the limited interest in some of these conversions,
little effort has been made to update them. In particular, data
including haploid loci\index{Haploid data} or in three-digits format may
not be converted into valid input for the other programs.

The following menu appears:

\begin{verbatim}
 File conversion (diploid data, 2-digits coding only):

      GENEPOP --> FSTAT (F statistics) ........................ 1
      GENEPOP --> BIOSYS (letter code) ........................ 2
      GENEPOP --> BIOSYS (number code) ........................ 3
      GENEPOP --> LINKDOS (D statistics) ...................... 4

      Main menu  .............................................. 5
\end{verbatim}

Sub-option 1 converts the Genepop input file into the format required by
the Fstat\index{Fstat program} program of Goudet (1995). The new format
is saved in the file \emph{yourdata}\texttt{.DAT}.

Sub-options 2 and 3 converts the Genepop input file into the format
required by Biosys (Swofford and Selander 1989),\index{Biosys program}
either the letter or the number code. The new format is saved in the
file \emph{yourdata}\texttt{.BIO}. You should add the STEP procedures at
the end of this new file before running Biosys. Refer to the Biosys
manual for details.

Sub-option 4 converts the Genepop input file into the format required by
Linkdos\index{Linkdos program}, a program described by Garnier-Géré and
Dillmann (1992) and based on Black and Krafsur (1985). This program
performs pairwise linkage disequilibria analyses in subdivided
populations and Ohta (1982)'s (1982) \(D\)
statistics.\index{Linkage disequilibrium!Ohta's statistics} The new
format is saved in the file \emph{yourdata}\texttt{.LKD}. The source
Linkdos program (LINKDOS.PAS) and an executable (LINKDOS.EXE) have been
distributed with previous versions of Genepop with permission of their
authors, and are still available on the
\href{http://kimura.univ-montp2.fr/~rousset/Genepop.htm}{Genepop
distribution page}. The executable distributed with Genepop has been
compiled for 40 samples, 20 loci and 99 alleles per locus. It may be
wise to relabel alleles (option 8.3) before the conversion. Garnier-Géré
and Dillmann (1992) should be cited whenever this program is used.

\section{Option 8: Null alleles and some input file
utilities}\label{option-8-null-alleles-and-some-input-file-utilities}

The following menu appears\footnote{Former sub-option 3 (erasing all
  temporary files) has been discarded.}

\begin{verbatim}
 Miscellaneous :
    Null allele: estimates of allele frequencies .......... 1
    Diploidisation of haploid data ........................ 2
    Relabeling alleles .................................... 3
    Conversion to individual data with population names ... 4
    Conversion to individual data with individual names ... 5
    Random sampling of haploid genotypes from diploid ones  6

    Main Menu   ........................................... 7
\end{verbatim}

\subsection{Sub-option 1: null alleles}\label{sub-option-1-null-alleles}

\index{Null alleles} This sub-option allows estimation of gene
frequencies when a null allele is present. Different methods are
available: maximum likelihood, maximum likelihood with genotyping
failure, and Brookfield's (1996) estimator, which differences are
explained in Section \ref{null-alleles}.\footnote{The last two methods
  are new to Genepop 4.0.}

Genepop takes the allele with the highest number for a given locus
\textbf{across all populations} as the null allele.\footnote{This is a
  notable difference from Genepop 3.4, where the allele with the highest
  number in each population was taken as the null allele in this
  population. Consequently, null allele estimation is now meaningful
  even if no null homozygote is observed in a given population. The
  output format has also been improved, compared to earlier versions of
  Genepop, with a more logical ordering of results (samples within loci)
  and a final locus by population table of estimated null allele
  frequencies.} For example, if you have 4 alleles plus a null allele, a
null homozygote individual should be indicated as e.g. \texttt{0505} or
\texttt{9999} in the input file.

The default estimation method is maximum likelihood, using the EM
algorithm of Dempster, Laird, and Rubin (1977). Apparent null genotypes
may also be due to nonspecific genotyping failures. Joint maximum
likelihood estimation of such failure rate (``\(\beta\)'') and of allele
frequencies is available through the setting
\texttt{NullAlleleMethod=ApparentNulls}. Finally, the estimator of
Brookfield (1996) is also available through the setting
\texttt{NullAlleleMethod=B96}.\index{NullAlleleMethod
 setting} Confidence intervals for null allele frequencies are computed
for each locus in each population. Their coverage probability can be
modified by the same setting \texttt{CIcoverage} as in options 6.5 and
6.6.\index{CIcoverage setting}

The output file is saved in the file \emph{yourdata}\texttt{.NUL}. This
file may contain

\begin{itemize}
\item
  For the maximum likelihood methods, estimated allelic frequencies and
  predicted numbers of homozygotes and of heterozygotes with a null
  allele. For example, in an output such as

\begin{verbatim}
 Allele   EM freq.  Homoz.    Null Heter.
  1      0.2762    2.7046     4.2954
  2      0.2576    1.8500     3.1500
  3      0.2251    1.3567     2.6433
  4      0.0217    0.0000     0.0000
 Null    0.2193
\end{verbatim}

  of the seven (\texttt{2.7046+4.2954}) apparent homozygotes for allele
  1, it is predicted that 4.2954 are actually heterozygotes for allele 1
  and for the null allele. This predicted value is the expected, or
  average, number of such heterozygotes over different samples with the
  same number of apparent genotypes, under the assumptions of the model.
\item
  a summary locus-by-population table of estimates of null allele
  frequencies.
\item
  a summary locus-by-population table of estimates of genotyping failure
  frequencies (``\texttt{beta}''), if applicable.
\item
  A table of bootstrap confidence intervals for estimates of null allele
  frequencies.
\end{itemize}

Note that there may be insufficient information to compute estimates
and/or confidence intervals: not enough alleles in the sample, for
example. These are indicated by the message \texttt{No\ information}.
Sometimes the point estimate can formally be computed but the computed
CI is not meaningful. This happens for example in case of heterozygote
excess, and generates a \texttt{(No\ info\ for\ CI)} warning (if all
pseudo-samples generated by some resampling technique show an
heterozygote excess, all pseudo-estimates of null allele frequency will
be zero and there is no information to construct a non-null CI from this
distribution).

The confidence intervals for null allele frequencies are obtained by a
bootstrap method, and are \textbf{not suitable} for testing for the
presence of null alleles, because the null hypothesis is at the boundary
of the parameter space (Andrews 2000). Instead, the exact score test for
Hardy-Weinberg proportions can be used.

\subsection{Sub-option 2: Diploidisation of haploid
data}\label{sub-option-2-diploidisation-of-haploid-data}

\index{Haploid data!to haploid} This sub-option ``diploidizes'' haploid
loci. For example, the line\\
\texttt{popul\ 1,\ 01\ 02\ 10\ 00}\\
of an haploid dataset with 4 loci, will become\\
\texttt{popul\ 1,\ 0101\ 0202\ 1010\ 0000}.\\
Only haploid data are thus modified in a mixed haploid/diploid data
file. The new file is named \texttt{D}\emph{yourdata}.\footnote{No
  longer truncated to 8 letters as it was in earlier versions of Genepop}

Note that there may no longer be any need for this option for further
analyses with Genepop (except perhaps as a preliminary to file
conversions, option 7), since Genepop 4.0 now perform analyses on
haploid data without such prior ``diploidization'' (don't forget the
\texttt{EstimationPloidy=Haploid} setting).

\subsection{Sub-option 3: Relabeling alleles
names}\label{sub-option-3-relabeling-alleles-names}

\index{Relabeling alleles} This sub-option relabels all alleles starting
from 1 up to \(x\), \(x\) being the true number of distinct alleles for
each locus. The new file is named \texttt{N}\emph{yourdata}. The
correspondence between the old and the new numbering is indicated in the
file \emph{new\_file\_name}.NUM. This option was originally introduced
in Genepop because for some options, the memory space required depends
on the highest allele number. I don't expect this to be a cause of
concern now.

\subsection{Sub-options 4 and 5: Conversion of population data to
individual
data}\label{sub-options-4-and-5-conversion-of-population-data-to-individual-data}

\index{Individual-based analysis!conversion of data for} These
sub-options convert ``population'' data (with several individuals per
\texttt{Pop} to ``individual'' data where each individual is put in a
distinct \texttt{Pop}. This is useful for individual-based analyses of
isolation by distance and, in this perspective, the name of each
individual is replaced by what should be its coordinates, that is,
either the name of the last individual in the original population
(sub-option 4), or the name of each individual if their locations are
distinguished (sub-option 5)\footnote{New to Genepop 4.3}.

\subsection{Sub-option 6: Random sampling of haploid genotypes from
diploid
ones}\label{sub-option-6-random-sampling-of-haploid-genotypes-from-diploid-ones}

This sub-option randomly samples haploid genotypes at diploid
loci.\footnote{New to Genepop 4.3} This may be useful for external
analyses that require haploid data or that would be biased by
Hardy-Weinberg disequilibria.

\chapter{Evaluating the performance of inferences for Isolation by
distance}\label{evaluating-the-performance-of-inferences-for-isolation-by-distance}

Genepop can analyze multiple files, using the settings settings

\begin{verbatim}
 GenepopRootFile=file                   <-- or GenepopRootFileName...
 JobMin=1
 JobMax=100
\end{verbatim}

\index{GenepopRootFile setting} \index{JobMin} \index{JobMax} This will
perform analysis of data in files \texttt{file}1 to \texttt{file}100.
Default values of these three settings are \texttt{GP}, 1, and 1. Users
need to assemble results from the multiple output files. A more
integrated output is provided for analyses of isolation by distance. For
the regression estimators of \(D\sigma^2\) (menu options 6.5 and 6.6),
the \texttt{result.CI} file will contain a table of point estimates,
bootstrap confidence intervals, and (if requested using the
\texttt{testPoint} setting) the bootstrap P-value for a given tested
neighborhood value. including the performance of the bootstrap
confidence intervals.

The \texttt{Performance=}\emph{value} setting\index{Performance setting}
provides a convenient (if somewhat ad hoc) shortcut for selecting the
following analyses:

\begin{longtable}[]{@{}ll@{}}
\toprule
analysis & \emph{value}\tabularnewline
\midrule
\endhead
\(\hat{a}\), 1-dim. & \texttt{aLinear} or equivalently
\texttt{a1D}\tabularnewline
\(\hat{e}\), 2-dim. & \texttt{aPlanar} or \texttt{a2D}\tabularnewline
\(\hat{a}\), 1-dim. & \texttt{eLinear} or \texttt{e1D}\tabularnewline
\(\hat{e}\), 2-dim. & \texttt{ePlanar} or \texttt{e2D}\tabularnewline
\(F/(1-F)\), 1-dim. & \texttt{FLinear} or \texttt{F2D}\tabularnewline
\(F/(1-F)\), 2-dim. & \texttt{FPlanar} or \texttt{F2D}\tabularnewline
\bottomrule
\end{longtable}

\texttt{Performance} sets Genepop in batch mode.\index{Batch mode} Then,
the \texttt{GenepopRootFile}, \texttt{JobMin}, and \texttt{JobMax}
values must be given in the settings file. Alternatively, these values
can be given interactively if the \texttt{Ask} or \texttt{Default} mode
\index{Mode
setting} has been specified \emph{after} the \texttt{Performance}
setting, in which case Genepop will carry all further computations in
\texttt{Default} mode.

\chapter{Methods}\label{methods}

This section is only intended as a quick reference guide. The primary
literature should be consulted for further information about the methods
implemented in Genepop.

\section{Null alleles}\label{null-alleles}

\index{Null alleles} When apparent null homozygotes are observed, one
may wonder whether these are truly null homozygotes, or whether some
technical failure independent of genotype has occurred. Maximum
likelihood estimates of null allele frequency, or of this frequency
jointly with the failure rate, can be obtained by the EM algorithm
(Dempster, Laird, and Rubin 1977; Hartl and Clark 1989; Kalinowski and
Taper 2006), which is one of the methods implemented in Genepop (menu
option 8.1).

Also implemented is a simpler estimator defined by Brookfield (1996) for
the case where apparent null homozygotes are true null homozygotes. He
also described this as a maximum likelihood estimator, but there are
some (often small) differences with the ML estimates derived by the EM
algorithm as implemented in this and previous versions of Genepop, which
may to be due to the fact that Brookfield wrote a likelihood formula for
the number of apparent homozygotes and heterozygotes, while the EM
implementation is based on a likelihood formula where apparent
homozygotes and heterozygotes for different alleles are distinguished.

For the case where one is unsure whether apparent null homozygotes are
true null homozygotes, Chakraborty et al. (1992) described a method to
estimate the null allele frequency from the other data, excluding any
apparent null homozygote. The estimator is not implemented in Genepop
because, beyond its relatively low efficiency, its behavior is sometimes
puzzling (for example, where there is no obvious heterozygote in a
sample, the estimated null allele frequency is always 1, whatever the
number of alleles obviously present and even if only non-null genotypes
are present). Actually, even if apparent null homozygotes are not true
null homozygotes, their number bring some information, and it is more
logical to estimate the null allele frequency jointly with the
nonspecific genotyping failure rate by maximum likelihood (Kalinowski
and Taper 2006). This analysis is possible when at least three alleles
are obviously present.

\section{Exact tests}\label{exact-tests}

The probability of a sample of genotypes depends on allele frequencies
at one or more loci. In the tests of Hardy Weinberg equilibrium,
population differentiation and pairwise independence between loci
(``linkage equilibrium'') implemented in Genepop, one is not interested
in the allele frequencies themselves and, given they are unknown, the
aim is to derive valid conclusions whatever their values. In these
different cases, this can be achieved by considering only the
probability of samples conditional\index{Exact tests!conditional tests}
on observed allelic (e.g.~for HW tests) or genotypic counts (e.g.~for
tests of population differentiation not assuming HW equilibrium).
Because exact probabilities are computed, these conditional tests are
also known as exact tests. See Cox and Hinkley (1974) and Lehmann (1994)
for the underlying theory; a much more elementary introduction to the
tests implemented in Genepop is Rousset and Raymond (1997).

\section{Algorithms for exact tests}\label{algorithms-for-exact-tests}

Conditional tests require in principle the complete enumeration of all
possible samples satisfying the given condition. In many cases this is
not practical, and the \(P\)-value may be computed by simple permutation
algorithms\index{Exact tests!permutation algorithms} or by more
elaborate Markov chain algorithms, in particular the Metropolis-Hastings
algorithm (Hastings
1970).\index{Exact tests!Metropolis-Hastings algorithm} The latter
algorithm explores the universe of samples satisfying the given
condition in a ``random walk'' fashion. For HW testing Guo and Thompson
(1992) found a Metropolis-Hastings algorithm to be efficient compared to
permutations. A slight modification of their algorithm is implemented in
Genepop. Guo and Thompson also considered tests for contingency tables
(Technical report No. 187, Department of Statistics, University of
Washington, Seattle, USA, 1989) and again a slightly modified algorithm
is implemented in Genepop (Raymond and Rousset 1995a). A run of the
Markov chain (MC) algorithms starts with a dememorization step; if this
step is long enough, the state of the chain at the end of the
dememorization is independent of the initial state. Then, further
simulation of the MC is divided in batches. In each batch a P-value
estimate is derived by counting the proportion of time the MC spends
visiting sample configurations more extreme (according to the given test
statistic) than the observed sample. If the batches are long enough, the
P-value estimates from successive batches are essentially independent
from each other and a standard error for the P-value can be derived from
the variance of per-batch P-values (Hastings 1970). As could be
expected, the longer the runs, the lower the standard error.

\section{Accuracy of P values estimated by the Markov chain
algorithms}\label{accuracy-of-p-values-estimated-by-the-markov-chain-algorithms}

\index{Markov chain algorithms!accuracy} For most data sets the MC
``mixes well'' so that the default values of the dememorization length
and batch length implemented in Genepop appear quite sufficient (in many
other applications of MC algorithms, things are not so simple; e.g.
Brooks and Gelman 1998). Nevertheless, inaccurate P-values can be
detected when the standard error is large, or else if the number of
switches (the number of times the sample configuration changes in the MC
run)\index{Markov chain algorithms!switches} is low (this may occur when
the P-value estimate is close to 0 or 1). Therefore, it is wise to
increase the number of batches if the standard error is too large, in
particular if it is of the order of \(P\) (the P-value) for small \(P\)
or of the order of \(1-P\) for large \(P\), or else if the number of
switches is low (\(<1000\)).

\section{Test statistics}\label{test-statistics}

The Markov chain algorithms were first implemented for probability
tests, i.e.~tests where the rejection zone is defined out of the least
likely samples under the null
hypothesis.\index{Exact tests!probability test} Such tests also had
Fisher's preference (e.g. Fisher 1935); in particular the probability
test for independence in contingency tables is known as Fisher's exact
test.\index{Exact tests!Fisher's} However, probability tests are not
necessarily the most powerful. Depending on the alternative hypothesis
of importance, other test statistics are often preferable (see again Cox
and Hinkley 1974 or Lehmann (1994) for textbook accounts). Efficient
tests for detecting heterozygote excesses and deficits (Rousset and
Raymond 1995) were introduced in Genepop from the start (see option 1),
and log likelihood ratio (\(G\)) tests were introduced with the
implementation of the genotypic tests for population differentiation
(Goudet et al. 1996). The allelic weighting implicit in the \(G\)
statistic is indeed optimal for detecting differentiation under an
island model (Rousset 2007) and use of the \(G\) statistic has been
generalized to all contingency table tests in Genepop 4.0, though
probability tests performed in earlier versions of Genepop are still
available.

Global tests are performed either using methods tuned to specific
alternative hypotheses (for heterozygote excess or deficiency) or using
Fisher's combination of probabilities technique. While the latter has
been criticized (Whitlock 2005), the recommended alternative can fail
spectacularly on discrete data.\index{Combination of different tests}

\section{\texorpdfstring{Estimating \(F\)-statistics and related
quantities}{Estimating F-statistics and related quantities}}\label{estimating-f-statistics-and-related-quantities}

The definition of
\(F\)-statistics\index{F-statistics@$F$-statistics!definition} used here
is

\[\begin{aligned}
 {F_\mathrm{IS}}\equiv &\frac{Q_1-Q_2}{1-Q_2}\\
 {F_\mathrm{ST}}\equiv &\frac{Q_2-Q_3}{1-Q_3}\\
 {F_\mathrm{IT}}\equiv &\frac{Q_1-Q_3}{1-Q_3}
 \end{aligned}\]

where the \(Q\) are probabilities of identity in state, \(Q_1\) among
genes (gametes) within individuals, \(Q_2\) among genes in different
individuals within groups (populations), and \(Q_3\) among groups
(populations). Such formulas appear in Cockerham and Weir (1987); see
François Rousset (2002a) for an account of most implications of such
definitions, except estimation.

The commonly held idea that it is more difficult to estimate
\(F\)-statistics when there are more alleles is generally incorrect;
actually many inferences may be more accurate when more alleles are
present (e.g. Leblois, Estoup, and Rousset 2003, at least as long as
gene diversity is less than 0.8). The issue is not to estimate the
frequencies of all alleles, but only to estimate the above
ratios.\index{F-statistics@$F$-statistics!estimation formulas} Any
expression of the form \((Q_i-Q_j)/(1-Q_j)\) can be estimated as
\((\hat{Q}_i-\hat{Q}_j)/(1-\hat{Q}_j)\) where any \(\hat{Q}_k\) is the
observed frequency of identical pairs of genes in the sample, among
pairs satisfying the condition designated by the \(k\) index. This is
only slightly different (see Rousset 2007) from what the following
estimators achieve.

\subsection{ANOVA estimators: single- and multilocus
definitions}\label{Fmulti}

Well-known work by Cockerham (e.g. Cockerham 1973; Weir and Cockerham
1984) has used the formalism of analysis of variance (ANOVA) to define
estimators of \(F\)-statistics. These estimators may be expressed in
terms of the mean sums of squares \(MSG\), \(MSI\), \(MSP\) (for
Gametes, Individuals, and Populations) computed by an analysis of
variance (see e.g. Weir 1996). Equivalently, they can be expressed in
terms of ``components of variances'' \(\hat{\sigma}^2_G\),
\(\hat{\sigma}^2_I\), \(\hat{\sigma}^2_P\) which are unbiased estimates
of the corresponding parametric ``components of variances''
\(\sigma^2_G\), \(\sigma^2_I\), \(\sigma^2_P\) in an ANOVA model. The
snag is, in general (and in some notable applications), these parametric
``components of variance'' are not variances but rather differences
between variances and can be negative. The \(\sigma^2\) notation is
misleading in this respect; this is a lasting source of confusion,
explained in Rousset (2007). Of course, the \(\hat{\sigma}^2\)
estimators can be negative even if the \(\sigma^2\) parameters are
positive, but this is a distinct issue.

The mean squares can themselves be interpreted in terms of observed
frequencies \(\hat{Q}\) of identical pairs of genes in the sample. For
balanced samples, the relationships are simple:

\(1-\hat{Q}_1=MSG\equiv \hat{\sigma}^2_G\),
\(\hat{Q}_1-\hat{Q}_2=(MSI-MSG)/2\equiv \hat{\sigma}^2_I\) and
\(\hat{Q}_2-\hat{Q}_3=(MSP-MSI)/(2n)\equiv \hat{\sigma}^2_P\) where
\(n\) is group size. Hence the single-group (single-population)
\(F_\mathrm{IS}\) estimator is

\[\label{}
    \frac{\hat{Q}_1-\hat{Q}_2}{1-\hat{Q}_2}=
    \frac{MSI-MSG}{MSI+MSG}=
    \frac{\hat{\sigma}^2_I}{\hat{\sigma}^2_I+\hat{\sigma}^2_G}.\]

For unbalanced groups (``populations'' of unequal size), estimates over
several groups are complex weighted averages of observed frequencies of
identical pairs of genes within groups, not detailed here (see Rousset
2007). However, ANOVA expressions still satisfy
\(MSG\equiv \hat{\sigma}^2_G\) and
\((MSI-MSG)/2\equiv \hat{\sigma}^2_I\), and
\((MSP-MSI)/(2n_c)\equiv \hat{\sigma}^2_P\) where \(n_c\) is a function
of the size of each group (\(n_c\equiv [S_1-S_2/S_1]/(n-1)\), where
\(S_1\) is the total sample size, \(S_2\) is the sum of squared group
sizes, and \(n\) is the number of non-empty groups). Then

\[\begin{gathered}
    \hat{F}_{\mathrm{IS}}=    \frac{MSI-MSG}{MSI+MSG}=
    \frac{\hat{\sigma}^2_I}{\hat{\sigma}^2_I+\hat{\sigma}^2_G}, \\
        \hat{F}_{\mathrm{ST}}=    \frac{MSP-MSI}{MSP+(n_c-1)MSI+n_cMSG}=
    \frac{\hat{\sigma}^2_P}{\hat{\sigma}^2_P+\hat{\sigma}^2_I+\hat{\sigma}^2_G}, \\
        \hat{F}_{\mathrm{IT}}=    \frac{MSP+(n_c-1)MSI-n_cMSG}{MSP+(n_c-1)MSI+n_cMSG}=
    \frac{\hat{\sigma}^2_P+\hat{\sigma}^2_I}{\hat{\sigma}^2_P+\hat{\sigma}^2_I+\hat{\sigma}^2_G}.\end{gathered}\]

With several loci, such an analysis is performed for each locus \(i\)
and the multilocus estimate is the ratio of a weighted sum of the above
locus-specific numerators over locus-specific denominators. However,
there is no single consistent way to compute the weighted sums. Weir and
Cockerham (1984)'s multilocus estimators are defined in terms of
intermediate statistics \(a\), \(b\), and \(c\) for each locus, which
appear to be the \(\hat{\sigma}^2\)'s. The numerator of the multilocus
estimator of \(F_\mathrm{ST}\) is thus
\(\sum_{\textrm{loci }i}a_i=\sum_{i}[(MSP-MSI)/(2n_c)]_i\). On the other
hand (Weir 1996's) multilocus estimators are defined from distinct
intermediate statistics \(S_1\), \(S_2\), and \(S_3\) for each locus,
where for locus \(i\), \(S_{1i}=[(MSP-MSI)]_i/(2\bar{n})\) for an
average sample size across loci \(\bar{n}\), and the numerator of the
multilocus estimate is
\(\sum_{\textrm{loci }i}S_i=\sum_{i}[a n_c]_i/\bar{n}\). Hence the 1984
and 1996 estimators slightly differ.

However, both give the same weight to the estimates of the \(Q\)'s for a
locus typed at 5 individuals in each subpopulation as for a locus typed
at 50 individuals in each subpopulation. Genepop follows another logic.
The multilocus estimator of \(F_\mathrm{ST}\) has numerator
\(\sum_i [n_c(MSP-MSI)]_i\), which will give 10 time more weight to the
\(Q\) estimates for the more intensively typed locus. `Explicit'
formulas for the estimators are:

\[\begin{gathered}
    \hat{F}_{\mathrm{IS}}=    \frac{\sum_i [n_c(MSI-MSG)]_i}{\sum_i [n_c(MSI+MSG)]_i}=
    \frac{\sum_i [n_c\hat{\sigma}^2_I]_i}{\sum_i [n_c\hat{\sigma}^2_I+n_c\hat{\sigma}^2_G]_i}, \\
        \hat{F}_{\mathrm{ST}}=    \frac{\sum_i [MSP-MSI]_i}{\sum_i [MSP+(n_c-1)MSI+n_cMSG]_i}=
    \frac{\sum_i [n_c\hat{\sigma}^2_P]_i}{\sum_i [n_c\hat{\sigma}^2_P+n_c\hat{\sigma}^2_I+n_c\hat{\sigma}^2_G]_i}, \\
        \hat{F}_{\mathrm{IT}}=    \frac{\sum_i [MSP+(n_c-1)MSI-n_cMSG]_i}{\sum_i [MSP+(n_c-1)MSI+n_cMSG]_i}=
    \frac{\sum_i [n_c\hat{\sigma}^2_P+n_c\hat{\sigma}^2_I]_i}{\sum_i [n_c\hat{\sigma}^2_P+n_c\hat{\sigma}^2_I+n_c\hat{\sigma}^2_G]_i}.\end{gathered}\]

Data from the example file \texttt{Fmulti.txt} (3 samples, 3 loci)
illustrate the difference between results obtained by the different
methods:

\begin{longtable}[]{@{}llll@{}}
\toprule
Estimate & \(F_\mathrm{IS}\) & \(F_\mathrm{ST}\) &
\(F_\mathrm{IT}\)\tabularnewline
\midrule
\endhead
locus 1 & -0.0483 & 0.5712 & 0.5505\tabularnewline
locus 2 & -0.1161 & 0.8560 & 0.8393\tabularnewline
locus 3 & 0.0051 & -0.0023 & 0.0028\tabularnewline
Multilocus (1984 a,b,c method) & -0.0286 & 0.5606 &
0.5480\tabularnewline
Multilocus (1996 S1,S2,S3 method) & -0.0286 & 0.5633 &
0.5508\tabularnewline
Multilocus (Genepop v3.3 and later) & -0.0275 & 0.5436 &
0.5310\tabularnewline
\bottomrule
\end{longtable}

Most of the time the different estimators yield close values; I expect
the Genepop method to provide better \(F_\mathrm{ST}\) estimates under
weak differentiation.

\subsection{\texorpdfstring{Microsatellite allele sizes,
\(R_\mathrm{ST}\), and
\(\rho_\mathrm{ST}\)}{Microsatellite allele sizes, R\_\textbackslash{}mathrm\{ST\}, and \textbackslash{}rho\_\textbackslash{}mathrm\{ST\}}}\label{rho-stats}

\index{Allele size-based statistics!Rst@$\Rst$}\index{Rst@$\Rst$|see{Allele size-based statistics}}
Following Slatkin (1995), statistics based on allele size have been
widely used. The parameters \(\rho_\mathrm{IS}\), \(\rho_\mathrm{ST}\)
and \(\rho_\mathrm{IT}\) and their
estimators\index{Allele size-based statistics!rhost@$\rho_{\mathrm{ST}}$}\index{rhost@$\rho_{\mathrm{ST}}$}
are defined by replacing any \(1-Q_k\) by the expected square difference
in allele size between the genes compared (Rousset 1996) in all formulas
above, and any \(1-\hat{Q}_k\) by the observed mean square difference
(more formulas are given in Michalakis and Excoffier 1996). Then the
estimators become plain ANOVA estimators of intraclass correlation for
allele size; if there are only two alleles,
\(\hat{\rho}_{\mathrm{ST}}=\hat{F}_{\mathrm{ST}}\), but Slatkin's
\(R_{\mathrm{ST}}\neq\hat{F}_{\mathrm{ST}}\).

\subsection{\texorpdfstring{Robertson and Hill's estimator of
\(F_\mathrm{IS}\)}{Robertson and Hill's estimator of F\_\textbackslash{}mathrm\{IS\}}}\label{robertson-and-hills-estimator-of-f_mathrmis}

This estimator, reported in options 1 and 5, was designed to have lower
variance than the ANOVA estimator and no small-sample bias when
\(F_\mathrm{IS}\) is low, assuming that deviations from Hardy-Weinberg
proportions are characterized by the same \(F_\mathrm{IS}\) for all
pairs of alleles (Robertson and Hill
1984).\index{\Fis!@Robertson \&\ Hill's estimator of \Fis} The score
test computed in heterozygote excess and deficiency sub-options of
option 1 is equivalent to this estimator for testing purposes.

\section{Bootstraps}\label{bootstraps}

\index{Confidence intervals!bootstrap} Option 6 constructs approximate
bootstrap confidence (ABC) intervals (DiCiccio and Efron 1996), assuming
that each locus is an independent realization of genealogical and
mutation processes. The bootstrap is a general methodology with
different incarnations. The ABC methods were chosen because they balance
moderate computation needs with good accuracy compared to alternatives.
Bootstrap methods are approximate, and simulation tests of their
performance (a too rare deed in statistical population genetics) for the
present application are reported in Leblois, Estoup, and Rousset (2003)
and Watts et al. (2007).

The ABC method is also applied over individuals in option 8 to compute
confidence intervals for null allele frequency estimates.

\section{Mantel test}\label{mantel-test}

\index{Mantel test} The principle of the Mantel permutation procedure is
to permute samples between geographical locations, so it generates a
distribution conditional on having \(n\) given sets of genotypic data in
\(n\) different samples. The permutations provide the distribution of
any statistic under the null hypothesis of independence between the two
variables (here, genotype counts and geographic location).

Mantel (1967) considered a particular statistics and approximations for
its distribution. Instead, Genepop uses no such approximation. Isolation
by distance will generate positive correlations between geographic
distance and genetic distance estimates, and this is best tested using
one-tailed P-values. The program provides both one-tailed P-values. The
probability of observing the sample correlation is the sum of these two
P-values minus 1.

\subsection{Misuse 1: tests of correlation at different
distance}\label{misuse-1-tests-of-correlation-at-different-distance}

Genetic processes of isolation by distance generate asymptotically
decreasing variation in genetic differentiation with increasing
geographic distances, and there is some temptation to use the Mantel
test to test for the presence of correlation at specific distances.
However, Genepop prevents this as this is logically unsound, and the
more quantitative methods it provides are better suited to address
variation of patterns with distance.

As soon as a process generates data with an expected non-zero
correlation at some distance, it contradicts the null hypothesis under
which the Mantel test is an exact test. Thus it may not make sense to
use a Mantel test for testing correlation at some distance if there is
correlation at another distance.

One can still wonder whether a permutation-based test could have some
approximate validity for testing absence of correlation at some
distance. However, the bootstrap procedure already addresses this case.
Alternative procedures would require further definition on an ad-hoc
basis to be operational (e.g., the idea of eliminating \emph{samples}
that form \emph{pairs} below or above a given distance may not
unambiguously define a \emph{sample} selection procedure that will
retain power) and would be likely to generate some confusion.

For these reasons, in the present implementation the Mantel tests are
always based on all pairs, ignoring all selection of data according to
distance.

\subsection{Misuse 2: partial Mantel
tests}\label{misuse-2-partial-mantel-tests}

Partial Mantel tests\index{Mantel test!partial} have been used to test
for effects of a variable Y on a response variable Z, while supposedly
removing spatial autocorrelation effects on Z. Both standard theory of
exact tests (as used by Raufaste and Rousset 2001) and simulation (Oden
and Sokal 1992; Raufaste and Rousset 2001; François Rousset 2002b;
Guillot and Rousset 2013) show that the permutation procedure of the
Mantel test is not appropriate for the partial Mantel test when the Y
variable itself presents spatial correlations. Asymptotic arguments have
also been proposed to support the use of such permutation tests (e.g.
Anderson 2001) but they fail in the same conditions. As shown by
Raufaste and Rousset (2001), the problem is inherent to the permutation
procedure, not to a specific test statistic. Unfortunately, some papers
maintain confusion about these different aspects of ``partial Mantel
tests''. Legendre and Fortin (2010) argued how miserable the papers by
Raufaste and Rousset (2001) and François Rousset (2002b) were, and
claimed that some versions of the tests should be preferred because they
used pivotal statistics (without evidence that the statistics were
indeed pivotal, a property that depends on the statistical model).
Guillot and Rousset (2013) reviewed old and more recent literature
demonstrating issues with the partial Mantel test, provided new
simulations showing that the different tests discussed by Legendre and
Fortin (2010) failed, and criticized their verbal arguments. Despite
this, Legendre, Fortin, and Borcard (2015) criticized this more recent
paper again for ignoring the old literature, and repeated the same kind
of verbal explanations that have previously failed.

\chapter{Code maintenance, credits, contact,
etc.}\label{code-maintenance-credits-contact-etc.}

\section{Code maintenance}\label{code-maintenance}

Distribution of Genepop as an R package means that the code is portable
to the major operating systems supported by R. New version are checked
using a variety of tools available in the R environment (including
valgrind and so-called sanitizers). Tests against more or less standard
examples from the literature are also applied. These tests can be found
in the \texttt{tests/testthat} directory of the distributed archive.

\section{Credits for the current
version}\label{credits-for-the-current-version}

The R package and the R markdown version of the documentation were
originally developed by Jimmy Lopez (Labex Cemeb) and Khalid Belkhir
(Institut des Sciences de l'Évolution) from the C++ sources and LaTeX
documentation of the Genepop executable version 4.6, and further
modified by F. Rousset.

\section{Previous history}\label{previous-history}

Version 4.0 of Genepop was a C++ rewrite of Genepop 3.4 (Raymond and
Rousset 1995b) by F.R., using draft C translations of many Genepop
modules by O. Guillaume, N. Benhamou and A. André, and some draft C++
classes by R. Leblois.

Beyond M. Raymond and F.R., credit for previous Genepop code is as
follows. The complete enumeration procedure for HW tests was derived
from Fortran code provided by E. J. Louis (Inst. Mol. Med., Oxford, UK).
Some of the procedures for isolation by distance ``between individuals''
were first written by R. Leblois with help from S. Piry (INRA-CBGP,
Montpellier). P. David, É. Imbert and S. Samadi wrote some early code in
1993.

B. Anderson, M.A.~Beaumont, A. Becher, T.J.C. Beebee, S. Bellman, L.
Bernatchez, D. Bourguet, J. Britton-Davidian, E. Bucheli, J. Carlier, G.
Carmody, R. Castilho, F. Catzeflis, C. Chevillon, J. Clayton, J. Dallas,
P. David, P. Dias, B. Dodd, R. Eritja, A. Estoup, A.-B. Failloux, E.
Fjerdingstad, R.C. Fleischer, A.J. Gharrett, S. T. Glenn, S.(?) Goodman,
J. Goudet, L. Henke, D. Innes, P. Jarne, L. Jermiin, J. Kelso, N.
Khromov-Borissov, J. Lagnel, M. Lascoux, L.S. Magnussen, J. Mallet, D.,
(?) McDonald, C. Moran, F. Nicholas, I. Olivieri, M. van Oppen, N.
Pasteur, R. Paxton, F. Renaud, H. Rosa, L., P. W. Shaw, Shapiro, J.
Shykoff, D. Sicard, J. Slate, M. Slatkin, M. Small, T. Staedler, F.
Thomas, F. Viard, P. Waldmann, K. J. Wetherall, (?) Winker, Z. Xu, made
suggestions or tests on the various states of Genepop until version 3.4.

T. Antão, E. Archer, R.I. Bailey, J.S.F. Barker, D. Bourguet, T. Devitt,
É. Imbert, R. Leblois, T. de Meeüs, P. Morin, S. Ponsard, V. Ravigné, E.
Taschen, and Y. Zimmermann have pointed issues or have stimulated
additional developments of more recent versions.

\section{Contact}\label{contact}

\index{Bug reports} If you think you have found a bug, you can contact
me. Requests which do not meet the following requirements are likely to
meet poor response. Please provide a minimal input file illustrating the
suspected problem, whenever relevant. Please use the latest version of
Genepop taken from a web page I maintain. \textbf{Note that I do not
maintain the ``Genepop on the web'' port of Genepop: any question
related to this port should be addressed to Eleanor Morgan.} Please
specify the version of Genepop you are using. Please do not ask whether
Genepop is commercial software. Please read this documentation.

I may answer queries about methods implemented in Genepop, and the more
so when they are specific to Genepop. But in most cases there are
published references describing the methods, cited in this
documentation. Please read this documentation.

\subsection{Bug fixes since release of Genepop version 3.4 in May 2003
until first release of Genepop
4.0:}\label{bug-fixes-since-release-of-genepop-version-3.4-in-may-2003-until-first-release-of-genepop-4.0}

\index{Bugs}\index{Bugs} The sign of the lower confidence interval bound
for regression slope in Isolde did not appear on output file when it was
negative.

For computation of allele size-based statistics (Option 6.2 and 6.4)
with the option ``allele name \emph{=} allele size'', the allele `99'
was interpreted as having size zero.

See the distribution page for more recent bug fixes.

\chapter{Copyright}\label{copyright}

All contents of the R package are covered by its license, the
GPL-compatible CeCill 2.1 license (see
\url{https://cecill.info/licences/Licence_CeCILL_V2.1-en.html}).

\index{Bootstrap|see{Confidence intervals}}
\index{Genepop@\Genepop, differences from previous versions|see{also footnotes throughout this document}}
\index{F-statistics@$F$-statistics|see{also \Fis}}
\index{Population differentiation|see{Differentiation}}
\index{Selecting subset of samples|see{Population type selection}}
\index{Input file|see{GenepopInputFile}}
\index{Heterozygosities|see{Gene diversities}}
\index{Maximum sample size|see{Maxima}}
\index{Exact tests|see{also Differentiation; Linkage disequilibrium; Hardy-Wein\-berg tests; Mantel test}}
\index{Data selection!by ploidy, see estimationPloidy}
\index{Data selection!subset of samples, see popTypeSelection}
\index{Hardy-Wein\-berg tests|optN{Option 1}}
\index{Hardy-Wein\-berg tests!multisample score test|optN{Options 1.4 \&\ 1.5}}
\index{Linkage disequilibrium|optN{Option 2}}
\index{Differentiation|optN{Option 3}}
\index{Private allele method|optN{Option 4}}
\index{Gene diversities|optN{Options 5.2 \&\ 5.3}}
\index{Fis@\Fis!per sample per locus|optN{Options 5.1 \&\ 5.2}}
\index{Fis@\Fis!multisample per locus|optN{Options 5.2 \&\ 6.1}}
\index{Fis@\Fis!per sample multilocus|optN{Option 5.2}}
\index{rhois@$\rho_{\mathrm{IS}}$!per sample per locus|optN{Option 5.3}}
\index{rhois@$\rho_{\mathrm{IS}}$!multisample per locus|optN{Option 5.3 \&\ 6.3}}
\index{rhois@$\rho_{\mathrm{IS}}$!per sample multilocus|optN{Option 5.3}}
\index{Fis@\Fis!multisample multilocus|optN{Option 6.1}}
\index{F-statistics@$F$-statistics!Fst@\Fst|optN{Options 6.1 \&\ 6.2}}
\index{rhois@$\rho_{\mathrm{IS}}$!multisample multilocus|optN{Option 6.3}}
\index{rhost@$\rho_{\mathrm{ST}}$|optN{Options 6.3 \&\ 6.4}}
\index{Allele size-based statistics|optN{Options 6.3 \&\ 6.4}}
\index{Mantel test|optN{Options 6.5 \&\ 6.6}}
\index{Isolation by distance!between individuals|optN{Option 6.5}}
\index{Isolation by distance!between groups|optN{Option 6.6}}
\index{File conversions|optN{Option 7}}
\index{Null alleles|optN{Option 8.1}}
\index{Relabeling alleles|optN{Option 8.3}}
\index{Individual data from population data|optN{Option 8.4}}

\printindex

\chapter*{Bibliography}\label{bibliography}
\addcontentsline{toc}{chapter}{Bibliography}

\hypertarget{refs}{}
\hypertarget{ref-Anderson01}{}
Anderson, Marti J. 2001. ``Permutation Tests for Univariate or
Multivariate Analysis of Variance and Regression.'' \emph{Can. J. Fish
Aquatic Scis.} 58: 626--39.

\hypertarget{ref-Andrews00}{}
Andrews, Donald W. K. 2000. ``Inconsistency of the Bootstrap When a
Parameter Is on the Boundary of the Parameter Space.''
\emph{Econometrica} 68 (2): 399--405.

\hypertarget{ref-BlackK85}{}
Black, W. C., IV, and E. S. Krafsur. 1985. ``A FORTRAN Program for the
Calculation and Analysis of Two-Locus Linkage Disequilibrium
Coefficients.'' \emph{Theor. Appl. Genetics} 70: 491--96.

\hypertarget{ref-Brookfield96}{}
Brookfield, J. F. Y. 1996. ``A Simple New Method for Estimating Null
Allele Frequency from Heterozygote Deficiency.'' \emph{Mol. Ecol} 5:
453--55.

\hypertarget{ref-BrooksG98}{}
Brooks, Stephen, and Andrew Gelman. 1998. ``General Methods for
Monitoring Convergence of Iterative Simulations.'' \emph{J. Comput.
Graphical Statistics} 7: 434--55.

\hypertarget{ref-ChakrabortyADB92}{}
Chakraborty, R., M. de Andrade, S.P. Daiger, and B. Budowle. 1992.
``Apparent Heterozygote Deficiencies Observed in DNA Typing Data and
Their Implications in Forensic Applications.'' \emph{Ann. Hum. Genetics}
56: 45--57.

\hypertarget{ref-Cockerham73}{}
Cockerham, C. Clark. 1973. ``Analyses of Gene Frequencies.''
\emph{Genetics} 74: 679--700.

\hypertarget{ref-CockerhamW87}{}
Cockerham, C. Clark, and Bruce S. Weir. 1987. ``Correlations, Descent
Measures: Drift with Migration and Mutation.'' \emph{PNAS} 84: 8512--4.

\hypertarget{ref-CoxH74}{}
Cox, D. R., and D. V. Hinkley. 1974. \emph{Theoretical Statistics}.
London: Chapman \& Hall.

\hypertarget{ref-DempsterLR77}{}
Dempster, A. P., N. M. Laird, and D. B. Rubin. 1977. ``Maximum
Likelihood from Incomplete Data via the \emph{EM} Algorithm (with
Discussion).'' \emph{JRSSB} 39: 1--38.

\hypertarget{ref-DiCiccioE96}{}
DiCiccio, Thomas J., and Bradley Efron. 1996. ``Bootstrap Confidence
Intervals (with Discussion).'' \emph{Stat. Sci.} 11: 189--228.

\hypertarget{ref-Phylip}{}
Felsenstein, J. 2005. ``PHYLIP (Phylogeny Inference Package) Version
3.6.''

\hypertarget{ref-Fisher35}{}
Fisher, Ronald Aylmer. 1935. ``The Logic of Inductive Inference (with
Discussion).'' \emph{JRSS} 98: 39--82.

\hypertarget{ref-GarnierD92}{}
Garnier-Géré, P., and C. Dillmann. 1992. ``A Computer Program for
Testing Pairwise Linkage Disequilibria in Subdivided Populations.''
\emph{J. Hered.} 83: 239.

\hypertarget{ref-Goudet95}{}
Goudet, J. 1995. ``FSTAT (Version 1.2): A Computer Program to Calculate
F-Statistics.'' \emph{J. Hered.} 86: 485--86.

\hypertarget{ref-GoudetRMR96}{}
Goudet, Jérome, Michel Raymond, Thierry de Meeüs, and François Rousset.
1996. ``Testing Differentiation in Diploid Populations.''
\emph{Genetics} 144: 1931--8.

\hypertarget{ref-GuillotR13}{}
Guillot, G., and François Rousset. 2013. ``Dismantling the Mantel
Tests.'' \emph{Methods Ecol. Evol.} 4: 336--44.

\hypertarget{ref-GuoT92}{}
Guo, Sun Wei, and Elizabeth A. Thompson. 1992. ``Performing the Exact
Test of Hardy-Weinberg Proportion for Multiple Alleles.''
\emph{Biometrics} 48: 361--72.

\hypertarget{ref-Haldane54}{}
Haldane, John Burdon Sanderson. 1954. ``An Exact Test for Randomness of
Mating.'' \emph{Journal of Genetics} 52: 631--35.

\hypertarget{ref-HartlC2e}{}
Hartl, Daniel L., and Andrew G. Clark. 1989. \emph{Principles of
Population Genetics}. Second. Sunderland, Mass.: Sinauer.

\hypertarget{ref-Hastings70}{}
Hastings, W. K. 1970. ``Monte Carlo Sampling Methods Using Markov Chains
and Their Applications.'' \emph{Biometrika} 57: 97--109.

\hypertarget{ref-KalinowskiT06}{}
Kalinowski, Steven T., and Mark L. Taper. 2006. ``Maximum Likelihood
Estimation of the Frequency of Null Alleles at Microsatellite Loci.''
\emph{Conserv. Genetics} 7: 991--95.

\hypertarget{ref-LebloisX14}{}
Leblois, R., P. Pudlo, J. Néron, F. Bertaux, C. R. Beeravolu, R.
Vitalis, and F Rousset. 2014. ``Maximum Likelihood Inference of
Population Size Contractions from Microsatellite Data.'' \emph{Mol.
Biol. Evol.} 31: 2805--23.

\hypertarget{ref-LebloisER03}{}
Leblois, Raphael, Arnaud Estoup, and François Rousset. 2003. ``Influence
of Mutational and Sampling Factors on the Estimation of Demographic
Parameters in a `Continuous' Population Under Isolation by Distance.''
\emph{Mol. Biol. Evol.} 20: 491--502.

\hypertarget{ref-LegendreF10}{}
Legendre, Pierre, and Marie-Josée Fortin. 2010. ``Comparison of the
Mantel Test and Alternative Approaches for Detecting Complex
Multivariate Relationships in the Spatial Analysis of Genetic Data.''
\emph{Mol. Ecol. Resources} 10 (5): 831--44.

\hypertarget{ref-LegendreFB15}{}
Legendre, Pierre, Marie-Josée Fortin, and Daniel Borcard. 2015. ``Should
the Mantel Test Be Used in Spatial Analysis?'' \emph{Methods Ecol.
Evol.} 6: 1239--47.

\hypertarget{ref-Lehmann94test}{}
Lehmann, E. L. 1994. \emph{Testing Statistical Hypotheses}. Second. New
York: Chapman \& Hall.

\hypertarget{ref-Levene49}{}
Levene, Howard. 1949. ``On a Matching Problem Arising in Genetics.''
\emph{Annals of Mathematical Statistics} 20: 91--94.

\hypertarget{ref-LoiselleSNG95}{}
Loiselle, Bette A., Victoria L. Sork, John Nason, and Catherine Graham.
1995. ``Spatial Genetic Structure of a Tropical Understory Shrub
\emph{Psychotria Officinalis} (Rubiaceae).'' \emph{Am. J. Bot.} 82:
1420--5.

\hypertarget{ref-LouisD87}{}
Louis, Edward J., and Everett R. Dempster. 1987. ``An Exact Test for
Hardy-Weinberg and Multiple Alleles.'' \emph{Biometrics} 43: 805--11.

\hypertarget{ref-Mantel67}{}
Mantel, Nathan. 1967. ``The Detection of Disease Clustering and a
Generalized Regression Approach.'' \emph{Cancer Research} 27: 209--20.

\hypertarget{ref-MehtaP83}{}
Mehta, Cyrus R., and Nitin R. Patel. 1983. ``A Network Algorithm for
Performing Fisher's Exact Test in \(r \times
	c\) Contingency Tables.'' \emph{JASA} 78: 427--34.

\hypertarget{ref-MichalakisE96}{}
Michalakis, Yannis, and Laurent Excoffier. 1996. ``A Generic Estimation
of Population Subdivision Using Distances Between Alleles with Special
Interest to Microsatellite Loci.'' \emph{Genetics} 142: 1061--4.

\hypertarget{ref-OdenS92}{}
Oden, Neal L., and Robert R. Sokal. 1992. ``An Investigation of
Three-Matrix Permutation Tests.'' \emph{J. Classif.} 9: 275--90.

\hypertarget{ref-Ohta82p}{}
Ohta, Tomoko. 1982. ``Linkage Disequilibrium Due to Random Genetic Drift
in Finite Subdivided Populations.'' \emph{PNAS} 79: 1940--4.

\hypertarget{ref-RaufasteR01}{}
Raufaste, Nathalie, and François Rousset. 2001. ``Are Partial Mantel
Tests Adequate?'' \emph{Evolution} 55: 1703--5.

\hypertarget{ref-RaymondR95evol}{}
Raymond, Michel, and François Rousset. 1995a. ``An Exact Test for
Population Differentiation.'' \emph{Evolution} 49: 1283--6.

\hypertarget{ref-RaymondR95}{}
---------. 1995b. ``GENEPOP Version 1.2: Population Genetics Software
for Exact Tests and Ecumenicism.'' \emph{J. Hered.} 86: 248--49.

\hypertarget{ref-RobertsonH84}{}
Robertson, Alan, and William G. Hill. 1984. ``Deviations from
Hardy-Weinberg Proportions: Sampling Variances and Use in Estimation of
Inbreeding Coefficients.'' \emph{Genetics} 107: 703--18.

\hypertarget{ref-Rousset96}{}
Rousset, François. 1996. ``Equilibrium Values of Measures of Population
Subdivision for Stepwise Mutation Processes.'' \emph{Genetics} 142:
1357--62.

\hypertarget{ref-Rousset97}{}
---------. 1997. ``Genetic Differentiation and Estimation of Gene Flow
from \(F\)-Statistics Under Isolation by Distance.'' \emph{Genetics}
145: 1219--28.

\hypertarget{ref-Rousset99g}{}
---------. 1999. ``Genetic Differentiation Within and Between Two
Habitats.'' \emph{Genetics} 151: 397--407.

\hypertarget{ref-Rousset00}{}
---------. 2000. ``Genetic Differentiation Between Individuals.''
\emph{J. Evol. Biol.} 13: 58--62.

\hypertarget{ref-Rousset02h}{}
---------. 2002a. ``Inbreeding and Relatedness Coefficients: What Do
They Measure?'' \emph{Heredity} 88: 371--80.

\hypertarget{ref-Rousset02e}{}
---------. 2002b. ``Partial Mantel Tests: Reply to Castellano and
Balletto.'' \emph{Evolution} 56: 1874--5.

\hypertarget{ref-Rousset07w}{}
---------. 2007. ``Inferences from Spatial Population Genetics.'' In
\emph{Handbook of Statistical Genetics}, edited by D. J. Balding, M.
Bishop, and C. Cannings, third, 945--79. Chichester, U.K.: Wiley.

\hypertarget{ref-RoussetL07}{}
Rousset, François, and Raphaël Leblois. 2007. ``Likelihood and
Approximate Likelihood Analyses of Genetic Structure in a Linear
Habitat: Performance and Robustness to Model Mis-Specification.''
\emph{Mol. Biol. Evol.} 24: 2730--45.

\hypertarget{ref-RoussetL12}{}
Rousset, François, and Raphaël Leblois. 2012. ``Likelihood-Based
Inferences Under Isolation by Distance: Two-Dimensional Habitats and
Confidence Intervals.'' \emph{Mol. Biol. Evol.} 29: 957--73.

\hypertarget{ref-RoussetR95}{}
Rousset, François, and Michel Raymond. 1995. ``Testing Heterozygote
Excess and Deficiency.'' \emph{Genetics} 140: 1413--9.

\hypertarget{ref-RoussetR97}{}
---------. 1997. ``Statistical Analyses of Population Genetic Data: Old
Tools, New Concepts.'' \emph{Tr. Ecol. Evol.} 12: 313--17.

\hypertarget{ref-Slatkin95}{}
Slatkin, Montgomery. 1995. ``A Measure of Population Subdivision Based
on Microsatellite Allele Frequencies.'' \emph{Genetics} 139: 457--62.

\hypertarget{ref-SwoffordS89}{}
Swofford, D. L., and R. B. Selander. 1989. \emph{BIOSYS-1. a Computer
Program for the Analysis of Allelic Variation in Population Genetics and
Biochemical Systematics. Release 1.7.} Champaign: Illinois Natural
History Survey.

\hypertarget{ref-VekemansH04}{}
Vekemans, X., and O. J. Hardy. 2004. ``New Insights from Fine-Scale
Spatial Genetic Structure Analyses in Plant Populations.'' \emph{Mol.
Ecol.} 13: 921--34.

\hypertarget{ref-WattsX07}{}
Watts, Phillip C., François Rousset, Ilik J. Saccheri, Raphaël Leblois,
Stephen J. Kemp, and David J. Thompson. 2007. ``Compatible Genetic and
Ecological Estimates of Dispersal Rates in Insect (\emph{Coenagrion
Mercuriale}: Odonata: Zygoptera) Populations: Analysis of `Neighbourhood
Size' Using a More Precise Estimator.'' \emph{Mol. Ecol.} 16: 737--51.

\hypertarget{ref-WeirC84}{}
Weir, B. S., and C. Clark Cockerham. 1984. ``Estimating \(F\)-Statistics
for the Analysis of Population Structure.'' \emph{Evolution} 38:
1358--70.

\hypertarget{ref-WeirbkII}{}
Weir, Bruce S. 1996. \emph{Genetic Data Analysis Ii}. Sunderland, Mass.:
Sinauer.

\hypertarget{ref-Whitlock05}{}
Whitlock, M. C. 2005. ``Combining Probability from Independent Tests:
The Weighted Z-Method Is Superior to Fisher's Approach.'' \emph{J. Evol.
Biol.} 18 (5): 1368--73.


\end{document}
